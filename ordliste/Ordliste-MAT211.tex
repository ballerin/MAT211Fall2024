\documentclass[11pt]{amsart}

%\usepackage[latin1]{inputenc}
\usepackage[utf8]{inputenc}


%\usepackage{array}
\newcommand{\mto}[1]{\stackrel{#1}\longrightarrow}
\newcommand{\mato}[1]{\stackrel{#1}\mapsto}
\newcommand{\lpil}{\longrightarrow}
\newcommand{\und}{\underline}



\begin{document}

%\baselineskip 15mmx

\vskip 3cm
\begin{center}{{\huge {Ordliste MAT211 Reell Analyse}}} \\
  \vskip 3mm
  For mer om norske matematiske fagtermer, se: \\
  https://matematikkradet.no/ordliste/
\end{center}


\vskip 5mm
\noindent{\bf }


\begin{tabbing}
  \= definisjonsområde/framengde AAAA \=  reduced echelon formAAA  \kill
  \>  {\bf Norsk:}  \>  {\bf Engelsk:} \\
\> absolutt integrerbar \> absolutely integrable \\
\> absolutt kontinuerleg \> absolutly continuous \\
\> alternerende rekke  \> alternating series  \\
\> avbildning \> map \\
\> begrenset\> bounded  \\
\> begrensning \> bound  \\
\> bilde \> image \\
\> blandet partiellderivert \> mixed partial derivative   \\
\> cauchyfølge \> Cauchy sequence  \\
\> overdekning \> covering  \\
\> overdekningsavbildning\> covering map  \\
\> delfølge \> subsequence  \\
\> derivert \> derivative  \\
\> ekstremalverdisetningen\> extreme value theorem \\
\> fikspunktteorem \> fixed point theorem  \\
\> følge  \> sequence  \\
\> grense \> limit  \\
\> ingensteds tett \> nowhere dense  \\
\> kjerne \> kernel  \\
\> kjerneregelen \> chain rule \\
\> kompletthet \> completeness  \\
\> konvergensintervall \> interval of convergence \\
\> kvotientrom \> quotient space  \\
\> kompact mengde \> compact set \\
\> konvergens \> convergence  \\
\> lineæroperator \> linear operator \\
\> lukket mengde \> closed set \\
\>  minste øvre skranke \> least upper bound  \\
\> metrisk rom \> metric space  \\
\> mellomverdisetningen /\\
\> skjæringssetningen \> intermediate value theore  \\
\> middelverdisetningen / \\
\> sekantsetningen \> mean value theorem  \\
\> nesten overalt \> almost everywhere \\
\> normert rom \> normed space \\
\>  partisjon\> partition \\ 
\> randbetingelse \> boundary condition  \\
\> skalarprodukt \> skalar product  \\
\> største nedre skranke \> greatest lowe bound  \\
\> skjevsymmetrisk lineæroperator \> skew-symmetric linear operator \\
\> tellbar \> countable \\
\> trigonometrisk polynom \> trigonometric polynomial  \\
\> ikke-tellbar / overtellbar \> uncountable  \\
\> tilordningsproblem \> assigment problem \\
\> totalt begrenset  \> totally bounded \\
\> ubegrenset \> unbounded  \\
\> urbilde \> preimage \\
\> \aa pen mengde \> open set \\

\end{tabbing}







%\> restklasse \> coset \\
%%  \> kjede \> chain \\
%%  \> stigende kjede \> ascending chain \\
%%  \> synkende kjede \> desecending chain \\
%%  \> betingelse \> condition
%\end{tabbing}
%
%
%\medskip \medskip
%\noindent{\bf Kapittel 7}
%
%\begin{tabbing}
%  \= definisjonsområde/framengde AAAA \=  reduced echelon formAAA  \kill
%  \>  {\bf Norsk:}  \>  {\bf Engelsk:} \\
%\> med klokken \> clockwise\\
%\> mot klokken \> counterclockwise\\
%%\> mot klokken \> counterclockwise
%\> abelsk \> abelian\\
%\> generell line\ae r gruppe \> general linear group\\
%\>  spesiell line\ae r gruppe\> special linear group\\
%\> additiv gruppe \> additive group\\
%\> ekte undergruppe \> proper subgroup\\
%\> syklisk gruppe \> cyclic group\\
%\> alternerende gruppe \> alternating group\\
%%  \> noethersk integritetsområde \> Noetherian integral domain
%\end{tabbing}
%
%\medskip \medskip
%\noindent{\bf Kapittel 8}
%
%
%\begin{tabbing}
%  \= definisjonsområde/framengde AAAA \=  reduced echelon formAAA  \kill
%  \>  {\bf Norsk:}  \>  {\bf Engelsk:} \\
%\> kvotientgruppe \> quotient group\\
%\> restklasse \> coset\\
%\> abelsk gruppe \> Abelian group\\
%\> endelig gruppe \> finite group\\
%\> indeks \> index\\
%\> faktorgruppe \> factor group\\
%\> kommutatorundergruppe \> commutator group\\
%%  \> noethersk integritetsområde \> Noetherian integral domain
%%  \> noethersk integritetsområde \> Noetherian integral domain
%
%%  \> noethersk integritetsområde \> Noetherian integral domain
%\end{tabbing}
%
%
%
%\medskip \medskip
%\noindent{\bf Kapittel 9}
%
%
%\begin{tabbing}
%  \= definisjonsområde/framengde AAAA \=  reduced echelon formAAA  \kill
%  \>  {\bf Norsk:}  \>  {\bf Engelsk:} \\
%\> torsjon gruppe \> torsion group \\
%\> nilpotent gruppe \> nilpotent group \\
%\> konjugert  \> conjugate \\
%\> sentralisator \> centralizer \\
%%  \> diskret valuasjon \> discrete valuation \\
%%  \> diskret valuasjon \> discrete valuation \\
%
%%  \> diskret valuasjon \> discrete valuation \\
%%  \> diskret valuasjonsring \> discrete valuation ring \\
%%  \> dedekindområde \> Dedekind domain \\
%\end{tabbing}
%
%\medskip \medskip
%\noindent{\bf Kapittel 10}
%
%
%\begin{tabbing}
%  \= definisjonsområde/framengde AAAA \=  reduced echelon formAAA  \kill
%  \>  {\bf Norsk:}  \>  {\bf Engelsk:} \\
%\> euklidsk omr\aa de  \> Euclidean domain \\
%\> hovedidealomr\aa de \> principal ideal domain \\
%\> entydig faktoriseringsomr\aa de  \> unique factorization domain \\
%%  \> gradert ring \> graded ring \\
%%
%%  \> homogen  \> homogeneous \\
%%  \> gradert ring \> graded ring \\
%%
%%  \> homogen  \> homogeneous \\
%%  \> gradert ring \> graded ring \\
%%
%%  \> homogen  \> homogeneous \\
%%  \> gradert ring \> graded ring \\
%\end{tabbing}
%
%\medskip \medskip
%\noindent{\bf Kapittel 11}
%
%
%\begin{tabbing}
%  \= definisjonsområde/framengde AAAA \=  reduced echelon formAAA  \kill
%  \>  {\bf Norsk:}  \>  {\bf Engelsk:} \\
%\>   kroppsutvidelse \> field extension  \\
%\> spennmengde/utspenningsmengde  \> spanning set \\
%\> vektorrom \> vector space\\
%\> line\ae rkombinasjon \> linear combination  \\
%\> line\ae rt uavhengig  \> linearly independent \\
%% \> regulær lokal ring \> regular local ring
%
%
%% \> poincarérekken \> Ponincaré series  \\
%% \> hilbertfunksjonen  \> Hilbert function \\
%% \> regulær lokal ring \> regular local ring
%\end{tabbing}
%
  
\end{document}

