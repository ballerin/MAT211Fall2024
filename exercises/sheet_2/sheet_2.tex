% !TeX program = latexmk -pdf -pdflatex="pdflatex -synctex=1 -interaction=nonstopmode -shell-escape" -jobname=% -pretex="\newcommand{\version}{noanswer}" -usepretex % | latexmk -pdf -pdflatex="pdflatex -synctex=1 -interaction=nonstopmode -shell-escape" -jobname=%_solutions -pretex="\newcommand{\version}{}" -usepretex % | txs:///view-pdf "?am)_solutions.pdf"

\documentclass[11pt]{article}% autres choix : report, book

% setting a default value in case it is compiled without the magic comment
\unless\ifdefined\version
\def\version{noanswer}
\fi

\usepackage[utf8]{inputenc}
\usepackage[T1]{fontenc}
\usepackage[english]{babel}
\usepackage{textcomp}
\usepackage{amsmath,amssymb,amsthm}
\usepackage{pxfonts}
\usepackage[a4paper]{geometry}
\usepackage{graphicx}
\usepackage{float}
\usepackage{xcolor}
\usepackage{microtype}
\usepackage{enumitem}
\usepackage{hyperref}
\usepackage{pgfplots}
\usepackage[\version]{exercise}
\hypersetup{pdfstartview=XYZ}% zoom par défaut
\newtheoremstyle{exercice}%
%\usepackage{tikz}% Faire figure, graphique...
{\topsep}% espace avant
{\topsep}% espace après
{\upshape}% police du corps du théorème
{}% indentation (vide pour rien, \parindent)
{\bfseries}% police du titre du théorème
{}% ponctuation après le théorème
{ }% espace après le titre du théorème (\newline = saut de ligne)
{\thmname{#1}\thmnumber{ \textup{#2}}. ---\thmnote{ \textnormal{\itshape#3.}}}% spécification
% du titre du théorème

%Custom symbols
\newcommand{\R}{\mathbb{R}}
\newcommand{\C}{\mathbb{C}}
\newcommand{\Q}{\mathbb{Q}}
\newcommand{\N}{\mathbb{N}}
\newcommand{\e}{\mathrm{e}}
\newcommand{\eps}{\varepsilon}

\newcommand{\st}{\;|\;}
\newcommand{\bigst}{\;\Bigg|\;}

\newtheorem{theorem}{Theorem}

%Customization of the Answers package
\def\AnswerName{Solution of exercise}
%\newcommand{\AnswerHeader}{\medskip{\textbf{ Answer of \ExerciseName\ \ExerciseHeaderNB}\smallskip}}
\renewcommand{\ExerciseHeader}{%
	\par\noindent
	\textbf{\large \ExerciseName\ \ExerciseHeaderNB \ExerciseHeaderTitle\ExerciseHeaderOrigin}%
	\par\nopagebreak\medskip
}

\renewcommand{\AnswerHeader}{%
	\par\noindent
	\textbf{\large Solution of \ExerciseName\ \ExerciseHeaderNB \ExerciseHeaderTitle}%
	\par\nopagebreak\medskip
}
\setlength{\ExerciseSkipAfter}{1\baselineskip}
\setlength{\AnswerSkipAfter}{1\baselineskip}

\title{MAT 211 : Exercise Sheet 2}
\author{Francesco Ballerin}
\date{\color{gray}{\small{francesco.ballerin@uib.no}}}

\pagestyle{empty}


\begin{document}
\begin{minipage}[t]{\dimexpr \textwidth-6cm-\columnsep}
     \maketitle
\end{minipage}
\hfill\noindent\raisebox{-1.5\height}{\includegraphics[scale=0.1]{../UiBlogoMN.png}}

\vspace{50pt}

\begin{Exercise}[title=*]
	Prove Proposition 1.15 of Rudin.
\end{Exercise}

\begin{Exercise}[title=*]
	Complete the rest of the proof of Proposition 1.18 of Rudin.
\end{Exercise}

\begin{Exercise}[title=*]
	Complete the rest of the proof of Theorem 1.4.5 of Abbott.
\end{Exercise}

\begin{Exercise}[title=*]
	Is the set of all irrational numbers countably infinite?
\end{Exercise}

\begin{Exercise}[title=**$\dagger$]
	Let $S$ be a set. The \textit{power set} of $S$, also denoted by $2^{S}$, is defined to be the set of all possible subsets of $S$. As an example, the power set of $\{0,1,2\}$ is $$\{\emptyset,\{0\},\{1\},\{2\},\{0,1\},\{0,2\},\{1,2\},\{1,2,3\}\}$$
	
	Consider now the power set of the natural numbers $2^\mathbb N$. What is the cardinality of this set? (i.e. is it countable or uncountable?).
	
	\bigskip
	
	[Hint: by contradiction assume it is countable and try to proceed in the same way as the proof for $\mathbb R$ uncountable]
\end{Exercise}

\begin{Exercise}[title=**]
	Let $X$ be an infinite set. Show that, for any $A\subset X$ finite, the sets $X$ and $X\setminus A$ have the same cardinality. 
\end{Exercise}

\begin{Exercise}[title=\text{**$\dagger$ - Hilbert's Hotel}]
	Hilbert's paradox of the Grand Hotel is a thought experiment that was introduced in a lecture by the famous mathematician David Hilbert to convey the notion of a set being either countable or uncountable. The idea is the follow:
	
	You are the manager at a Grand Hotel. Actually, the Grandest hotel on earth. It is so Grand it has an infinite amount of rooms, all numbered with a number in $\mathbb N$ (room 0, room 1, room 2, ...). 
	
	\begin{itemize}
		\item One day you find yourself with all rooms fully booked when a new customer shows up and ask if you can accomodate him at your hotel. Can you do that? (Mathematically, is the cardinality of $\mathbb N$ the same as $\mathbb N$ with an added value? Show a mapping between the 2 sets if it is or explain why it is not.)
		
		\item What if multiple new customers, in a finite amount (say $n$), show up? Can you accomodate them at your hotel? (Mathematically, is the cardinality of $\mathbb N$ the same as $\mathbb N$ with $n$ added values?)
		
		\item Suppose now a bus with a countable amount of new customers shows up. Can you accomodate all the new customers?
		
		\item Suppose now a countable amount of busses, each with a countable amount of new customers show up. Can you accomodate all the new customers?
		
		\item \textit{The problem can be extended as many (finite) layers one desires. }
		
	\end{itemize}
	This problem is a nice example of how statements like "there is a guest to every room" and "no more guests can be accommodated" are intuitively equivalent in the finite case but become not equivalent when there are (countably) infinite many rooms. 
	
\end{Exercise}

\begin{Exercise}[title=*]
	A set $S\subset\R$ is said to be bounded if $\exists M$ s.t. $\forall x\in S$ it holds that $|x|\leq M$.
	
	Is the union of two bounded sets a bounded set? What about the union of countably infinite bounded sets? 
\end{Exercise}

\begin{Exercise}[title=***]
	A real number $\R$ is said to be \emph{algebraic} if there exist integers $a_0, a_1, a_2,\dots, a_n\in\mathbb{Z}$, not all zero, s.t. 
	\[a_n x^n + a_{n-1}x^{n-1}+\cdots+ a_1x + a_0 =0\]
	i.e. if it is the root of a non-zero polynomial. If a real number is not algebraic then it is said to be \emph{transcendental}.
	\begin{enumerate}
		\item Show that all natural numbers are algebraic.
		\item Show that $\sqrt 2$ and $\sqrt 3$ are algebraic.
		\item Show that $\sqrt2 + \sqrt3$ is algebraic.
		\item For $n\in\N$ let $A_n$ be the sets of roots of polynomials of degree $n$ (any plynomial of degree $n$). It is a known fact that a specific polynomial of degree $n$ has at most $n$ roots (exactly $n$ if we consider the complex numbers rather than the real numbers). Is $A_n$ countable?
		\item Is the set of all algebraic numbers countable?
		\item What about the set of all transcendental numbers?
	\end{enumerate}
\end{Exercise}

\begin{Exercise}[title = ***]
	Exercise 1.5.11 of Abbott.
\end{Exercise}

\begin{Exercise}[title=**$\dagger$]
	Verify, using the definition of convergence of a sequence (i.e. given an $\epsilon$ find the corresponding $N\in\N$), that the following sequences converge to the proposed limit:
	\begin{enumerate}
		\item $\lim \frac{1}{\sqrt n} = 0$
		\item $\lim \frac{n+1}{n} = 1$
		\item $\lim \frac{2n+1}{5n+4} = \frac{2}{5}$
		\item $\lim \frac{2n^2}{n^3+3} = 0$
		\item $\lim \frac{\sin(n^2)}{\sqrt[3]{n}} = 0$
	\end{enumerate}
\end{Exercise}


\end{document}

