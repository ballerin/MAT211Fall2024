% \textsc{}!TeX program = latexmk -pdf -pdflatex="pdflatex -synctex=1 -interaction=nonstopmode -shell-escape" -jobname=% -pretex="\newcommand{\version}{noanswer}" -usepretex % | latexmk -pdf -pdflatex="pdflatex -synctex=1 -interaction=nonstopmode -shell-escape" -jobname=%_solutions -pretex="\newcommand{\version}{}" -usepretex % | txs:///view-pdf "?am)_solutions.pdf"

\documentclass[11pt]{article}% autres choix : report, book

% setting a default value in case it is compiled without the magic comment
\unless\ifdefined\version
\def\version{noanswer}
\fi

\usepackage[utf8]{inputenc}
\usepackage[T1]{fontenc}
\usepackage[english]{babel}
\usepackage{textcomp}
\usepackage{amsmath,amssymb,amsthm}
\usepackage{pxfonts}
\usepackage[a4paper]{geometry}
\usepackage{graphicx}
\usepackage{float}
\usepackage{xcolor}
\usepackage{microtype}
\usepackage{enumitem}
\usepackage{hyperref}
\usepackage{pgfplots}
\usepackage[\version]{exercise}
\hypersetup{pdfstartview=XYZ}% zoom par défaut
\newtheoremstyle{exercice}%
%\usepackage{tikz}% Faire figure, graphique...
{\topsep}% espace avant
{\topsep}% espace après
{\upshape}% police du corps du théorème
{}% indentation (vide pour rien, \parindent)
{\bfseries}% police du titre du théorème
{}% ponctuation après le théorème
{ }% espace après le titre du théorème (\newline = saut de ligne)
{\thmname{#1}\thmnumber{ \textup{#2}}. ---\thmnote{ \textnormal{\itshape#3.}}}% spécification
% du titre du théorème

%Custom symbols
\newcommand{\R}{\mathbb{R}}
\newcommand{\C}{\mathbb{C}}
\newcommand{\Q}{\mathbb{Q}}
\newcommand{\N}{\mathbb{N}}
\newcommand{\e}{\mathrm{e}}
\newcommand{\eps}{\varepsilon}

\newcommand{\st}{\;|\;}
\newcommand{\bigst}{\;\Bigg|\;}

\newtheorem{theorem}{Theorem}

%Customization of the Answers package
\def\AnswerName{Solution of exercise}
%\newcommand{\AnswerHeader}{\medskip{\textbf{ Answer of \ExerciseName\ \ExerciseHeaderNB}\smallskip}}
\renewcommand{\ExerciseHeader}{%
	\par\noindent
	\textbf{\large \ExerciseName\ \ExerciseHeaderNB \ExerciseHeaderTitle\ExerciseHeaderOrigin}%
	\par\nopagebreak\medskip
}

\renewcommand{\AnswerHeader}{%
	\par\noindent
	\textbf{\large Solution of \ExerciseName\ \ExerciseHeaderNB \ExerciseHeaderTitle}%
	\par\nopagebreak\medskip
}
\setlength{\ExerciseSkipAfter}{1\baselineskip}
\setlength{\AnswerSkipAfter}{1\baselineskip}

\title{MAT 211 : Exercise Sheet 13}
\author{Francesco Ballerin}
\date{\color{gray}{\small{francesco.ballerin@uib.no}}}

\pagestyle{empty}


\begin{document}
\begin{minipage}[t]{\dimexpr \textwidth-6cm-\columnsep}
     \maketitle
\end{minipage}
\hfill\noindent\raisebox{-1.5\height}{\includegraphics[scale=0.1]{../UiBlogoMN.png}}

\vspace{50pt}

\begin{Exercise}[title={*}]

\begin{enumerate}
	\item Let $f_n$ a sequence uniformly convergent to $f$. Show for $-\infty<a<b<\infty$ that  $$\displaystyle\lim_{n\rightarrow\infty} \int_{a}^{b} f_n=\int_a^b f$$  
	\item What if the integral is over the whole real line? Give an example where $f_n$ converge uniformly to $f$ but $\displaystyle\lim_{n\to\infty}\int_{-\infty}^{\infty}f_n \neq \int_{-\infty}^{\infty}f$.
\end{enumerate}
\end{Exercise}
\bigskip

\begin{Exercise}[title={*}]
Let $$\displaystyle I_n=\int_{1}^2 e^{-nx^2} dx.$$ Admit that directly computing $I_n$ is a challenging task, but that it is not difficult to compute the quantity $$\lim_{n\to\infty} I_n.$$
\end{Exercise}
\bigskip

\begin{Exercise}[title={**$\dagger$}]
\begin{enumerate}
	\item Prove that the sequence of functions $$f_n=\frac{1}{x+n}$$ converges uniformly on intervals of the type $[0,b]$ for $b>0$.\\Does the same result hold for $[0,+\infty)$?
	\item Prove that the sequence of functions $$f_n=\frac{nx}{nx+1}$$ converges pointwise but not uniformly on $[0,1]$.
\end{enumerate}
\end{Exercise}

\bigskip

\begin{Exercise}[title={**$\dagger$}]
\begin{enumerate}
	\item Prove that the series of functions $$\sum_1^{+\infty}\frac{\sin(nx)}{n^2}$$ converges absolutely and uniformly on $\mathbb{R}$.
	\item Prove that the series of functions $$\sum_1^{+\infty}(-1)^n\frac{x^2+n}{n^2}$$ converges uniformly in every bounded interval (say any interval of the type $[a,b]$ with $a<b$) but does not converge absolutely for any value of $x$.
\end{enumerate}
\end{Exercise}

\bigskip

\begin{Exercise}[title={**}]
Let
$$f_n(x)=
\begin{cases}
	0&(x<\frac{1}{n+1})\\
	(\sin\frac{\pi}{x})^2&(\frac{1}{n+1}\leq x \leq \frac{1}{n})\\
	0&(\frac{1}{n}<x)\\
\end{cases}$$

be a sequence of functions defined for $n\geq 1$. Study the continuity of $f_n$. Show that the sequence $(f_n)_{n\in\mathbb{N}}$ converges pointwise to a continuous function. Show that however the convergence is only pointwise and not uniform.
\end{Exercise}

\bigskip
\bigskip

\begin{Exercise}[title={*}]
Suppose $f\geq0$ is a continuous function on $[a,b]$. Show that $\int_a^b f(x)dx=0$ implies that $\forall x\in[a,b]\;\;f(x)=0$.

\bigskip
[Hint: proceed by contradiction]
\end{Exercise}


\bigskip

\begin{Exercise}[title={**}]
Let $f$ be a continuous function on $[0,1]$ and if $\forall n\in \mathbb{N}$
$$\int_0^1 f(x) x^n dx =0$$
and show that $f(x)=0$ on $[0,1]$.

\bigskip

[Hint: This means that the integral of $f$ multiblied by any polynomial is zero. Use the Stone-Weierstrass theorem to show that $\int_0^1 f^2(x)dx=0$]
\end{Exercise}

\end{document}

