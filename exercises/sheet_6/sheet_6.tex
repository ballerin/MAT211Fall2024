% !TeX program = latexmk -pdf -pdflatex="pdflatex -synctex=1 -interaction=nonstopmode -shell-escape" -jobname=% -pretex="\newcommand{\version}{noanswer}" -usepretex % | latexmk -pdf -pdflatex="pdflatex -synctex=1 -interaction=nonstopmode -shell-escape" -jobname=%_solutions -pretex="\newcommand{\version}{}" -usepretex % | txs:///view-pdf "?am)_solutions.pdf"

\documentclass[11pt]{article}% autres choix : report, book

% setting a default value in case it is compiled without the magic comment
\unless\ifdefined\version
\def\version{noanswer}
\fi

\usepackage[utf8]{inputenc}
\usepackage[T1]{fontenc}
\usepackage[english]{babel}
\usepackage{textcomp}
\usepackage{amsmath,amssymb,amsthm}
\usepackage{pxfonts}
\usepackage[a4paper]{geometry}
\usepackage{graphicx}
\usepackage{float}
\usepackage{xcolor}
\usepackage{microtype}
\usepackage{enumitem}
\usepackage{hyperref}
\usepackage{pgfplots}
\usepackage[\version]{exercise}
\hypersetup{pdfstartview=XYZ}% zoom par défaut
\newtheoremstyle{exercice}%
%\usepackage{tikz}% Faire figure, graphique...
{\topsep}% espace avant
{\topsep}% espace après
{\upshape}% police du corps du théorème
{}% indentation (vide pour rien, \parindent)
{\bfseries}% police du titre du théorème
{}% ponctuation après le théorème
{ }% espace après le titre du théorème (\newline = saut de ligne)
{\thmname{#1}\thmnumber{ \textup{#2}}. ---\thmnote{ \textnormal{\itshape#3.}}}% spécification
% du titre du théorème

%Custom symbols
\newcommand{\R}{\mathbb{R}}
\newcommand{\C}{\mathbb{C}}
\newcommand{\Q}{\mathbb{Q}}
\newcommand{\N}{\mathbb{N}}
\newcommand{\e}{\mathrm{e}}
\newcommand{\eps}{\varepsilon}

\newcommand{\st}{\;|\;}
\newcommand{\bigst}{\;\Bigg|\;}

\newtheorem{theorem}{Theorem}

%Customization of the Answers package
\def\AnswerName{Solution of exercise}
%\newcommand{\AnswerHeader}{\medskip{\textbf{ Answer of \ExerciseName\ \ExerciseHeaderNB}\smallskip}}
\renewcommand{\ExerciseHeader}{%
	\par\noindent
	\textbf{\large \ExerciseName\ \ExerciseHeaderNB \ExerciseHeaderTitle\ExerciseHeaderOrigin}%
	\par\nopagebreak\medskip
}

\renewcommand{\AnswerHeader}{%
	\par\noindent
	\textbf{\large Solution of \ExerciseName\ \ExerciseHeaderNB \ExerciseHeaderTitle}%
	\par\nopagebreak\medskip
}
\setlength{\ExerciseSkipAfter}{1\baselineskip}
\setlength{\AnswerSkipAfter}{1\baselineskip}

\title{MAT 211 : Exercise Sheet 6}
\author{Francesco Ballerin}
\date{\color{gray}{\small{francesco.ballerin@uib.no}}}

\pagestyle{empty}


\begin{document}
\begin{minipage}[t]{\dimexpr \textwidth-6cm-\columnsep}
     \maketitle
\end{minipage}
\hfill\noindent\raisebox{-1.5\height}{\includegraphics[scale=0.1]{../UiBlogoMN.png}}

\vspace{50pt}

\begin{Exercise}[title=**$\dagger$]
Show that the following sequences converge:
$$\begin{aligned}
	(a_n)_{n\in\mathbb{N}} &\text{ with } &a_n&=(1+\frac{1}{2})(1+\frac{1}{4})\ldots(1+\frac{1}{2^n})\\
	(b_n)_{n\in\mathbb{N}} &\text{ with } &b_1&= \sqrt 2 \\
	& &b_2  &=\sqrt{2+\sqrt 2}\\
	& &b_n &=\sqrt{2+\sqrt{2+\ldots+\sqrt 2}}\\
	(c_n)_{n\in\mathbb{N}} &\text{ with } &c_n &= 1+\frac{1}{2^2}+\frac{1}{3^2}+\ldots+\frac{1}{n^2}
\end{aligned}
$$

[Hint: Prove that the sequences $(a_n)_{n=1}^{\infty}$ and $(b_n)_{n=1}^{\infty}$ are monotonic and bounded. Show that $(c_n)_{n=1}^{\infty}$ is a Cauchy sequence using the inequality $\frac{1}{n^2}<\frac{1}{n-1}-\frac{1}{n}$, $n\in\mathbb N$.]
\end{Exercise}

\begin{Exercise}[title=*$\dagger$]
Recall that the partial limits of a given sequence are the limits of subsequences. Find all the partial limits for following sequences:
\begin{itemize}
	\item[i)]{$$\frac{1}{2},\frac{1}{2},\frac{1}{4},\frac{3}{4},\frac{1}{8},\frac{7}{8},\ldots,\frac{1}{2^n},\frac{2^n-1}{2^n},\ldots$$}
	\item[ii)]{$$1,\frac{1}{2},1+\frac{1}{2},\frac{1}{3},1+\frac{1}{3},\frac{1}{2}+\frac{1}{3},\frac{1}{4},1+\frac{1}{4},\frac{1}{2}+\frac{1}{4},\frac{1}{3}+\frac{1}{4},$$ $$\frac{1}{5}\ldots,\frac{1}{n},1+\frac{1}{n},\frac{1}{2}+\frac{1}{n},\ldots,\frac{1}{n-1}+\frac{1}{n},\frac{1}{n+1},\ldots$$}
\end{itemize}
\end{Exercise}

\begin{Exercise}[title= *]  
Give an example of a sequence $(x_n)_{n\in\mathbb{N}}$ in $\mathbb R$ such that a finite set of numers $$a_1,a_2,\ldots,a_p$$ are partial limits for $(x_n)_{n\in\mathbb{N}}$.
\end{Exercise}

\begin{Exercise}[title= **]
Give an example of a sequence $(x_n)_{n\in\mathbb{N}}$ in $\mathbb R$ such that a countably infinite set of numers $$a_1,a_2,\ldots,a_p, \ldots$$ are partial limits for $(x_n)_{n\in\mathbb{N}}$. What are other partial limits does the sequence $(x_n)_{n\in\mathbb{N}}$ admit?
\end{Exercise}

\begin{Exercise}[title=**]
	Review the proof of Theorem 3.17 of Rudin, which says that for a sequence $(s_n)_n$ of real numbers and $E$ the set of all the limits of subsequences $(s_{n_k})_k$ of $(s_n)_n$, $\sup E$ has the following properties:
	\begin{enumerate}
		\item $\sup E \in E$
		\item if $x>\sup E$ then there is a natural number $N\in\N$ such hat for $n\geq N$ it holds that $S_n<x$
	\end{enumerate}
\end{Exercise}

\begin{Exercise}[title=*$\dagger$]
	Show that for wo sequences $(s_n)_n$ and $(t_n)_n$ in $\R$, if $s_n\leq t_n$ for $n\geq N$ where $N$ is a fixed natural number, then
	$$\liminf_{n\to\infty} s_n \leq \liminf_{n\to\infty} t_n$$
	$$\limsup_{n\to\infty} s_n \leq \limsup_{n\to\infty} t_n$$
\end{Exercise}

\begin{Exercise}[title= **]
Let $(x_n)_{n\in\mathbb{N}}$ and $(y_n)_{n\in\mathbb{N}}$ be (not necessarily converging) sequences in $\mathbb{R}$.
\begin{itemize}
	\item[i)] Show that $$\liminf\limits_{n\to\infty}x_n+\liminf\limits_{n\to\infty}y_n\leq\liminf\limits_{n\to\infty}(x_n+y_n)\leq \liminf\limits_{n\to\infty}x_n+\limsup\limits_{n\to\infty}y_n$$
	\item[ii)] Show that $$\liminf\limits_{n\to\infty}x_n+\limsup\limits_{n\to\infty}y_n\leq\limsup\limits_{n\to\infty}(x_n+y_n)\leq \limsup\limits_{n\to\infty}x_n+\limsup\limits_{n\to\infty}y_n$$
	\item[iii)] Construct examples that make the previous inequalities strict.
\end{itemize}
\end{Exercise}


\begin{Exercise}[title=**$\dagger$]
	[Rudin 3.4]
Find the $\limsup$ and $\liminf$ for each one of the following sequences:
\begin{enumerate}
	\item[i)] $u_n=(-1)^n$,
	\item[ii)] $u_n=n\cos(\frac{n\pi}{4})$,
	\item[iii)] $u_1=0$; $u_{2n}=\frac{u_{2n-1}}{2}$; $u_{2n+1}=\frac{1}{2} + u_{2n}$,
	\item[iv)] $u_n=\sin(n)$
\end{enumerate}
\end{Exercise}

\begin{Exercise}[title= **$\dagger$]
Let $(p_n)_{n\in\mathbb{N}}$ and $(q_n)_{n\in\mathbb{N}}$ be Cauchy sequences in a metric space $(X,d)$. Let $(d_n)_{n\in\mathbb{N}}$ be the sequence of distances between $p_n$ and $q_n$, i.e.  $d_n=d(p_n,q_n)$. Show that $(d_n)_{n\in\mathbb{N}}$ converges in $\mathbb R$.
\end{Exercise}

\end{document}

