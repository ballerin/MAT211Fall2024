% !TeX program = latexmk -pdf -pdflatex="pdflatex -synctex=1 -interaction=nonstopmode -shell-escape" -jobname=% -pretex="\newcommand{\version}{noanswer}" -usepretex % | latexmk -pdf -pdflatex="pdflatex -synctex=1 -interaction=nonstopmode -shell-escape" -jobname=%_solutions -pretex="\newcommand{\version}{}" -usepretex % | txs:///view-pdf "?am)_solutions.pdf"

\documentclass[11pt]{article}% autres choix : report, book

% setting a default value in case it is compiled without the magic comment
\unless\ifdefined\version
\def\version{noanswer}
\fi

\usepackage[utf8]{inputenc}
\usepackage[T1]{fontenc}
\usepackage[english]{babel}
\usepackage{textcomp}
\usepackage{amsmath,amssymb,amsthm}
\usepackage{pxfonts}
\usepackage[a4paper]{geometry}
\usepackage{graphicx}
\usepackage{float}
\usepackage{xcolor}
\usepackage{microtype}
\usepackage{enumitem}
\usepackage{hyperref}
\usepackage{pgfplots}
\usepackage[\version]{exercise}
\hypersetup{pdfstartview=XYZ}% zoom par défaut
\newtheoremstyle{exercice}%
%\usepackage{tikz}% Faire figure, graphique...
{\topsep}% espace avant
{\topsep}% espace après
{\upshape}% police du corps du théorème
{}% indentation (vide pour rien, \parindent)
{\bfseries}% police du titre du théorème
{}% ponctuation après le théorème
{ }% espace après le titre du théorème (\newline = saut de ligne)
{\thmname{#1}\thmnumber{ \textup{#2}}. ---\thmnote{ \textnormal{\itshape#3.}}}% spécification
% du titre du théorème

%Custom symbols
\newcommand{\R}{\mathbb{R}}
\newcommand{\C}{\mathbb{C}}
\newcommand{\Q}{\mathbb{Q}}
\newcommand{\N}{\mathbb{N}}
\newcommand{\e}{\mathrm{e}}
\newcommand{\eps}{\varepsilon}

\newcommand{\st}{\;|\;}
\newcommand{\bigst}{\;\Bigg|\;}

\newtheorem{theorem}{Theorem}

%Customization of the Answers package
\def\AnswerName{Solution of exercise}
%\newcommand{\AnswerHeader}{\medskip{\textbf{ Answer of \ExerciseName\ \ExerciseHeaderNB}\smallskip}}
\renewcommand{\ExerciseHeader}{%
	\par\noindent
	\textbf{\large \ExerciseName\ \ExerciseHeaderNB \ExerciseHeaderTitle\ExerciseHeaderOrigin}%
	\par\nopagebreak\medskip
}

\renewcommand{\AnswerHeader}{%
	\par\noindent
	\textbf{\large Solution of \ExerciseName\ \ExerciseHeaderNB \ExerciseHeaderTitle}%
	\par\nopagebreak\medskip
}
\setlength{\ExerciseSkipAfter}{1\baselineskip}
\setlength{\AnswerSkipAfter}{1\baselineskip}

\title{MAT 211 : Exercise Sheet 7}
\author{Francesco Ballerin}
\date{\color{gray}{\small{francesco.ballerin@uib.no}}}

\pagestyle{empty}


\begin{document}
\begin{minipage}[t]{\dimexpr \textwidth-6cm-\columnsep}
     \maketitle
\end{minipage}
\hfill\noindent\raisebox{-1.5\height}{\includegraphics[scale=0.1]{../UiBlogoMN.png}}

\vspace{50pt}

\begin{Exercise}[title=**$\dagger$] Use the definition of the convergent series to show that the following series converge. Find their sums. $$\sum_{n=1}^{\infty}\frac{(-1)^{n-1}}{2^{n-1}}=1-\frac{1}{2}+\frac{1}{4}-\frac{1}{8}+\ldots+\frac{(-1)^{n-1}}{2^{n-1}}+\ldots$$ $$\sum_{n=1}^{\infty}\frac{2n-1}{2^n}=\frac{1}{2}+\frac{3}{2^2}+\frac{5}{2^3}+\ldots+\frac{2n-1}{2^n}+\ldots$$ $$\sum_{n=1}^{\infty}\frac{1}{n(n+1)}=\frac{1}{1\cdot 2}+\frac{1}{2\cdot 3}+\frac{1}{3\cdot 4}+\ldots+\frac{1}{n(n+1)}+\ldots$$
\end{Exercise}

\begin{Exercise}[title=*] Study the convergence of the series $\sum_{n=1}^{\infty}\sin (nx)$ for every $x\in\mathbb R^n$. 
\end{Exercise}


[Hint: Start fixing an arbitrary $x\in\mathbb R$ and study the convergence of the series by taking into consideration that for $x\neq \pi k$, $k\in\mathbb Z$, one can not have $\lim\limits_{n\to\infty}\sin(nx)=0$.]
\bigskip

\begin{Exercise}[title=**$\dagger$]
	Which of the following series converge and which diverge? Explain. $$\sum_{n=0}^{\infty}(-1)^n=1-1+1-1+\ldots$$ $$\sum_{n=1}^{\infty}\frac{1}{n!}=\frac{1}{1!}+\frac{1}{2!}+\frac{1}{3!}+\ldots+\frac{1}{n!}+\ldots$$ $$\sum_{n=1}^{\infty}\frac{1}{n\sqrt{n+1}}=\frac{1}{\sqrt 2}+\frac{1}{2\sqrt 3}+\frac{1}{3\sqrt 4}+\ldots+\frac{1}{n\sqrt{n+1}}+\ldots$$ $$\sum_{n=1}^{\infty}\frac{1}{\sqrt{(2n-1)(2n+1)}}=\frac{1}{\sqrt {1\cdot 3}}+\frac{1}{\sqrt {3\cdot 5}}+\ldots+\frac{1}{\sqrt{(2n-1)(2n+1)}}+\ldots$$
\end{Exercise}

\begin{Exercise}[title=**] Let $\sum_{n=1}^{\infty} a_n$ be a series such that $\forall n\;\; a_n>0$. Show that if the limit \begin{equation*}\label{A}\lim\limits_{n\to\infty}\frac{a_{n+1}}{a_n}=q\end{equation*} exists, then the limit \begin{equation*}\label{B}\lim\limits_{n\to\infty}\sqrt[n]{a_n}=q\end{equation*} also exists.\\ Is the inverse statement true?
\bigskip

[Hint: consider $\sum_{n=1}^{\infty}\frac{3+(-1)^n}{2^{n+1}}$.]
\end{Exercise}

\begin{Exercise}[title=**$\dagger$] Find the radius of convergence of each of the following power series: $$\sum_{n=1}^{\infty}n^3z^n,\qquad\sum_{n=1}^{\infty}\frac{2^n}{n!}z^n,\qquad\sum_{n=1}^{\infty}\frac{2^n}{n^2}z^n,\qquad\sum_{n=1}^{\infty}\frac{n^3}{3^n}z^n.$$
\end{Exercise}

\begin{Exercise}[title=**] Prove that the sequence $(x_n)_n$ for
	$$x_n=1+\frac{1}{2}+\frac{1}{3}+\ldots+\frac{1}{n}-\ln n,\quad n\in\mathbb N \backslash \{0\}$$ converges to a certain constant $\gamma$, which is known as the Euler constant.
\bigskip

[Hint: try to show that the sequence is both bounded and decreasing. To show it is decreasing, recall that $ln(1-x)$ is a concave function. To  show it is bounded consider $\int_1^{n+1}\frac{1}{t}dt = ln(n+1)$, which sits between the harmonic series and $ln(n)$]


\end{Exercise}

\begin{Exercise}[title=*] Find the value of the sum $$\sum_{n=1}^{\infty}\frac{(-1)^{n-1}}{n}=1-\frac{1}{2}+\frac{1}{3}-\frac{1}{4}+\frac{1}{5}-\frac{1}{6}+\ldots.$$

[Hint: try to apply two times the formula from the previous problem]
\end{Exercise}


\begin{Exercise}[title={**}] Define the number $e\in \mathbb R$ by the sum of the series
$$
e=\sum_{n=0}^{\infty}\frac{1}{n!}.
$$
\begin{enumerate}
	\item Show that this is a good definition (the series converges in $\mathbb{R}$) [Rudin 3.30].
	\item {Show that $e$ is an irrational number [Rudin 3.31].}
	\item {Show that $e$ is also the limit of the sequence $\Big(1+\frac{1}{n}\Big)^n$ [Rudin 3.32].}
\end{enumerate}

\end{Exercise}

\end{document}

