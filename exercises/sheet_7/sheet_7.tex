% !TeX program = latexmk -pdf -pdflatex="pdflatex -synctex=1 -interaction=nonstopmode -shell-escape" -jobname=% -pretex="\newcommand{\version}{noanswer}" -usepretex % | latexmk -pdf -pdflatex="pdflatex -synctex=1 -interaction=nonstopmode -shell-escape" -jobname=%_solutions -pretex="\newcommand{\version}{}" -usepretex % | txs:///view-pdf "?am)_solutions.pdf"

\documentclass[11pt]{article}% autres choix : report, book

% setting a default value in case it is compiled without the magic comment
\unless\ifdefined\version
\def\version{noanswer}
\fi

\usepackage[utf8]{inputenc}
\usepackage[T1]{fontenc}
\usepackage[english]{babel}
\usepackage{textcomp}
\usepackage{amsmath,amssymb,amsthm}
\usepackage{pxfonts}
\usepackage[a4paper]{geometry}
\usepackage{graphicx}
\usepackage{float}
\usepackage{xcolor}
\usepackage{microtype}
\usepackage{enumitem}
\usepackage{hyperref}
\usepackage{pgfplots}
\usepackage[\version]{exercise}
\hypersetup{pdfstartview=XYZ}% zoom par défaut
\newtheoremstyle{exercice}%
%\usepackage{tikz}% Faire figure, graphique...
{\topsep}% espace avant
{\topsep}% espace après
{\upshape}% police du corps du théorème
{}% indentation (vide pour rien, \parindent)
{\bfseries}% police du titre du théorème
{}% ponctuation après le théorème
{ }% espace après le titre du théorème (\newline = saut de ligne)
{\thmname{#1}\thmnumber{ \textup{#2}}. ---\thmnote{ \textnormal{\itshape#3.}}}% spécification
% du titre du théorème

%Custom symbols
\newcommand{\R}{\mathbb{R}}
\newcommand{\C}{\mathbb{C}}
\newcommand{\Q}{\mathbb{Q}}
\newcommand{\N}{\mathbb{N}}
\newcommand{\e}{\mathrm{e}}
\newcommand{\eps}{\varepsilon}

\newcommand{\st}{\;|\;}
\newcommand{\bigst}{\;\Bigg|\;}

\newtheorem{theorem}{Theorem}

%Customization of the Answers package
\def\AnswerName{Solution of exercise}
%\newcommand{\AnswerHeader}{\medskip{\textbf{ Answer of \ExerciseName\ \ExerciseHeaderNB}\smallskip}}
\renewcommand{\ExerciseHeader}{%
	\par\noindent
	\textbf{\large \ExerciseName\ \ExerciseHeaderNB \ExerciseHeaderTitle\ExerciseHeaderOrigin}%
	\par\nopagebreak\medskip
}

\renewcommand{\AnswerHeader}{%
	\par\noindent
	\textbf{\large Solution of \ExerciseName\ \ExerciseHeaderNB \ExerciseHeaderTitle}%
	\par\nopagebreak\medskip
}
\setlength{\ExerciseSkipAfter}{1\baselineskip}
\setlength{\AnswerSkipAfter}{1\baselineskip}

\title{MAT 211 : Exercise Sheet 7}
\author{Francesco Ballerin}
\date{\color{gray}{\small{francesco.ballerin@uib.no}}}

\pagestyle{empty}


\begin{document}
\begin{minipage}[t]{\dimexpr \textwidth-6cm-\columnsep}
     \maketitle
\end{minipage}
\hfill\noindent\raisebox{-1.5\height}{\includegraphics[scale=0.1]{../UiBlogoMN.png}}

\vspace{50pt}

\begin{Exercise}[title=**] Show that the following series converge. Find their sums. $$\sum_{n=1}^{\infty}\frac{(-1)^{n-1}}{2^{n-1}}=1-\frac{1}{2}+\frac{1}{4}-\frac{1}{8}+\ldots+\frac{(-1)^{n-1}}{2^{n-1}}+\ldots$$ $$\sum_{n=1}^{\infty}\frac{2n-1}{2^n}=\frac{1}{2}+\frac{3}{2^2}+\frac{5}{2^3}+\ldots+\frac{2n-1}{2^n}+\ldots$$ $$\sum_{n=1}^{\infty}\frac{1}{n(n+1)}=\frac{1}{1\cdot 2}+\frac{1}{2\cdot 3}+\frac{1}{3\cdot 4}+\ldots+\frac{1}{n(n+1)}+\ldots$$
\end{Exercise}

\begin{Exercise}[title=*] Study the convergence of the series $\sum_{n=1}^{\infty}\sin (nx)$ for every $x\in\mathbb R^n$. 
\end{Exercise}


[Hint: Start fixing an arbitrary $x\in\mathbb R$ and study the convergence of the series by taking into consideration that for $x\neq \pi k$, $k\in\mathbb Z$, one can not have $\lim\limits_{n\to\infty}\sin(nx)=0$.]
\bigskip

\begin{Exercise}[title=**$\dagger$]
	Which of the following series converge and which diverge? Explain. 
	$$\sum_{n=0}^{\infty}(-1)^n=1-1+1-1+\ldots$$ $$\sum_{n=1}^{\infty}\frac{1}{n!}=\frac{1}{1!}+\frac{1}{2!}+\frac{1}{3!}+\ldots+\frac{1}{n!}+\ldots$$ $$\sum_{n=1}^{\infty}\frac{1}{\sqrt{(2n-1)(2n+1)}}=\frac{1}{\sqrt {1\cdot 3}}+\frac{1}{\sqrt {3\cdot 5}}+\ldots+\frac{1}{\sqrt{(2n-1)(2n+1)}}+\ldots$$
\end{Exercise}

\begin{Exercise}[title=**$\dagger$] Find the radius of convergence of each of the following power series: $$\sum_{n=1}^{\infty}n^3z^n,\qquad\sum_{n=1}^{\infty}\frac{2^n}{n!}z^n,\qquad\sum_{n=1}^{\infty}\frac{2^n}{n^2}z^n,\qquad\sum_{n=1}^{\infty}\frac{n^3}{3^n}z^n.$$
\end{Exercise}



\begin{Exercise}[title=**$\dagger$]
	Suppose $f,g:\mathbb R \rightarrow \mathbb R $ are well defined real valued functions such that $\lim_{x\rightarrow a} f(x) =K$ and $\lim_{x\rightarrow a} g(x) = L$ exist. Let $c\in\mathbb R$ be a constant. Prove the following claims: 
	\begin{enumerate}
		\item $$\lim_{x\rightarrow a} cf(x) = c \lim_{x\rightarrow a} f(x) = cK$$
		\item $$\lim_{x\rightarrow a} f(x) + g(x) = \lim_{x \rightarrow a} f(x) + \lim_{x \rightarrow a} g(x) = K+L$$
		\item $$\lim_{x\rightarrow a} f(x) \cdot g(x) = \lim_{x \rightarrow a} f(x) \cdot \lim_{x \rightarrow a} g(x) = K\cdot L$$
		\item If $L\neq0$ then
		$$\lim_{x\rightarrow a} \frac{f(x)}{g(x)} = \frac{\lim_{x \rightarrow a} f(x) }{ \lim_{x \rightarrow a} g(x)} = \frac{K}{L}$$
		\item For $n\in\mathbb{N}$ $$\lim_{x\rightarrow a} [f(x)]^n = [\lim_{x\rightarrow a} f(x)]^n=K^n$$
	\end{enumerate}
	[Hint: recall a similar proposition with limit of sequences]
\end{Exercise}

\begin{Exercise}[title=**$\dagger$]
	Study uniform continuity of the following functions.
	\begin{enumerate}
		\item{$f(x)=\ln x$ on $(0,1)$,}
		\item{$f(x)=\frac{\sin x}{x}$ on $(0,\pi)$,}
		\item{$f(x)=\sqrt x$ on $[1,+\infty)$,}
	\end{enumerate}
\end{Exercise}

\begin{Exercise}[title=**$\dagger$]
	Study the following sequences of functions and determine if they diverge, if they converge pointwise, or converge uniformly.
	\begin{enumerate}
		\item $f_n(x) = \frac{x}{n}$ with $n\geq 1$.
		\item $f_n(x) = x^n$ with $n\geq 0$.
		\item $f_n(x) = e^{-x}\left(1+\frac{x}{n}\right)^{n}$ with $n\geq 1$.
	\end{enumerate}
\end{Exercise}

\begin{Exercise}[title=*$\dagger$] Draw the graphs of the following functions and study the continuity and discontinuity on their domain:
	\begin{itemize}
		\item[a)]{$f(x)=\sin\frac{\pi}{x}$ if $x\neq 0$ and $f(0)$ is arbitrary,}
		\item[b)]{$f(x)=x\sin\frac{\pi}{x}$ if $x\neq 0$ and $f(0)=0$,}
		\item[c)]{$f(x)=e^{-\frac{1}{x^2}}$,}
		\item[d)]{$f(x)=\left\{\begin{array}{lllc}\frac{1}{n} \quad & \text{if}\quad & x=\frac{m}{n},\\
				0 &\text{if}  & x\quad\text{is irrational}.\end{array}\right.
			$}
	\end{itemize}
\end{Exercise}

\begin{Exercise}[title=*]
	Is the sum of two functions $f(x)+g(x)$ discontinuous at $x_0$ if 
	\begin{itemize}
		\item[a)]{the function $f$ is continuous but the function $g$ is discontinuous at $x_0$?}
		\item[b)]{both of the functions $f$ and $g$ are discontinuous at $x_0$?}
	\end{itemize} Prove your result or construct counterexamples.
\end{Exercise}

\begin{Exercise}[title=*]
	Is the product of two functions $f(x)g(x)$ discontinuous at $x_0$ if 
	\begin{itemize}
		\item[a)]{the function $f$ is continuous but the function $g$ is discontinuous at $x_0$?}
		\item[b)]{both of the functions $f$ and $g$ are discontinuous at $x_0$?}
	\end{itemize} Prove your result or construct counterexamples.
\end{Exercise}



\begin{Exercise}[title=**]
	Prove that the sum and product of two uniformly continuous functions defined on an interval $(a,b)$, $-\infty<a<b<+\infty$ are uniformly continuous on the same interval.
\end{Exercise}

\begin{Exercise}[title=*]
	Is it true that if $f(x)$ is differentiable on the open interval $(a,b)$, with $-\infty<a<b<+\infty$, and $$\lim_{x\to a}f^{\prime}(x)=\infty$$ then $$\lim_{x\to a}f(x)=\infty?$$ 
	
	[Hint: Consider the example $f(x)=\sqrt[3]{x}$ as $x\to 0$]
\end{Exercise}

\begin{Exercise}[title=**]
	Use the mean value theorems to show the following inequalities:
	\begin{itemize}
		\item[a)]{$|\sin x-\sin y|\leq |x-y|$}
		\item[b)]{$py^{p-1}(x-y)\leq x^p-y^p\leq px^{p-1}(x-y)$\ \  if $0<y<x$ and $p>1$}
		\item[c)]{$\frac{a-b}{a}<\ln\frac{a}{b}<\frac{a-b}{b}$\ \  if $0<b<a$}
	\end{itemize} 
\end{Exercise}

\newpage

\begin{Exercise}[title=**]
	Use the derivative to show the following inequalities:
	\begin{itemize}
		\item[a)]{$e^x>1+x$ if $x\neq 0$}
		\item[b)]{$x-\frac{x^2}{2}<\ln(1+x)<x$ for $x>0$}
		\item[c)]{$(x^a+y^a)^{\frac{1}{a}}>(x^b+y^b)^{\frac{1}{b}}$ for $x>0$, $y>0$, and $0<a<b$.}
	\end{itemize} 
	
	[Hint: composition by monotonically increasing functions preserve inequalities]
	
	[Hint: if two functions have the same value at a point, what can we say from their derivatives?]
\end{Exercise}

\end{document}

