% !TeX program = latexmk -pdf -pdflatex="pdflatex -synctex=1 -interaction=nonstopmode -shell-escape" -jobname=% -pretex="\newcommand{\version}{noanswer}" -usepretex % | latexmk -pdf -pdflatex="pdflatex -synctex=1 -interaction=nonstopmode -shell-escape" -jobname=%_solutions -pretex="\newcommand{\version}{}" -usepretex % | txs:///view-pdf "?am)_solutions.pdf"

\documentclass[11pt]{article}% autres choix : report, book

% setting a default value in case it is compiled without the magic comment
\unless\ifdefined\version
\def\version{noanswer}
\fi

\usepackage[utf8]{inputenc}
\usepackage[T1]{fontenc}
\usepackage[english]{babel}
\usepackage{textcomp}
\usepackage{amsmath,amssymb,amsthm}
\usepackage{pxfonts}
\usepackage[a4paper]{geometry}
\usepackage{graphicx}
\usepackage{float}
\usepackage{xcolor}
\usepackage{microtype}
\usepackage{enumitem}
\usepackage{hyperref}
\usepackage{pgfplots}
\usepackage[\version]{exercise}
\hypersetup{pdfstartview=XYZ}% zoom par défaut
\newtheoremstyle{exercice}%
%\usepackage{tikz}% Faire figure, graphique...
{\topsep}% espace avant
{\topsep}% espace après
{\upshape}% police du corps du théorème
{}% indentation (vide pour rien, \parindent)
{\bfseries}% police du titre du théorème
{}% ponctuation après le théorème
{ }% espace après le titre du théorème (\newline = saut de ligne)
{\thmname{#1}\thmnumber{ \textup{#2}}. ---\thmnote{ \textnormal{\itshape#3.}}}% spécification
% du titre du théorème

%Custom symbols
\newcommand{\R}{\mathbb{R}}
\newcommand{\C}{\mathbb{C}}
\newcommand{\Q}{\mathbb{Q}}
\newcommand{\N}{\mathbb{N}}
\newcommand{\e}{\mathrm{e}}
\newcommand{\eps}{\varepsilon}

\newcommand{\st}{\;|\;}
\newcommand{\bigst}{\;\Bigg|\;}

\newtheorem{theorem}{Theorem}

%Customization of the Answers package
\def\AnswerName{Solution of exercise}
%\newcommand{\AnswerHeader}{\medskip{\textbf{ Answer of \ExerciseName\ \ExerciseHeaderNB}\smallskip}}
\renewcommand{\ExerciseHeader}{%
	\par\noindent
	\textbf{\large \ExerciseName\ \ExerciseHeaderNB \ExerciseHeaderTitle\ExerciseHeaderOrigin}%
	\par\nopagebreak\medskip
}

\renewcommand{\AnswerHeader}{%
	\par\noindent
	\textbf{\large Solution of \ExerciseName\ \ExerciseHeaderNB \ExerciseHeaderTitle}%
	\par\nopagebreak\medskip
}
\setlength{\ExerciseSkipAfter}{1\baselineskip}
\setlength{\AnswerSkipAfter}{1\baselineskip}

\title{MAT 211 : Exercise Sheet 4}
\author{Francesco Ballerin}
\date{\color{gray}{\small{francesco.ballerin@uib.no}}}

\pagestyle{empty}


\begin{document}
\begin{minipage}[t]{\dimexpr \textwidth-6cm-\columnsep}
     \maketitle
\end{minipage}
\hfill\noindent\raisebox{-1.5\height}{\includegraphics[scale=0.1]{../UiBlogoMN.png}}

\vspace{50pt}

\begin{Exercise}[title=*$\dagger$]
	Let $x,y\in \mathbb R$. Define the following functions:
	$$d_1(x,y)=(x-y)^2$$
	$$\quad d_2(x,y)=|x^2-y^2|$$
	$$\quad d_3(x,y)=\frac{|x-y|}{1+|x-y|}$$
	Determine, for each one of these, whether it is a metric or not.\\
	\smallskip
\end{Exercise}
\begin{Exercise}[title=*]
	Suppose that $(X,d)$ is a metric space. Is $\delta=\frac{d}{1+d}$ a distance?
\end{Exercise}

\begin{Exercise}[title=** - $\ell_p$ spaces]
	Let us consider the set $\mathbb R^n$ and the functions $\rho_p\colon \mathbb R^n\times \mathbb R^n\to \mathbb R$ defined by $$\rho_p(x,y)=(\sum_{k=1}^{n}|x_k-y_k|^p)^{1/p},\quad 1\leq p<\infty.$$ Show that $(\mathbb R^n, \rho_p)$ are metric spaces.
	
	Hint:
	\begin{enumerate}[label={\alph*)}]
		% \item [1.]{Assume the H\"{o}lder inequality. {\it Let $1<p<\infty$ and $q$ is defined by $1/p+1/q=1$}. Given $x,y\in\mathbb R^n$ we have $$\sum_{k=1}^{n}|x_ky_k|\leq (\sum_{k=1}^{n}|x_k|^p)^{1/p}(\sum_{k=1}^{n}|y_k|^q)^{1/q}.$$}
		\item {Assume the Minkowski inequality to be true:\\{\it Let $1<p<\infty$. If $x\in\mathbb R^n$ then we denote $|x|_p=(\sum_{k=1}^{n}|x_k|^p)^{1/p}$ (it is called $p$-norm of $x\in \mathbb R^n$).  Then if $x,y\in\mathbb R^n$ it follows that $$|x+y|_p\leq|x|_p+|y|_p.$$}}
		\item {Use this inequality to prove the triangle inequality for $\rho_p$.}
	\end{enumerate}
\end{Exercise}

\begin{Exercise}[title=*$\dagger$]
	Let the function $\rho_{max}(x,y)$ for $x,y\in\mathbb R^n$, $n\in\mathbb N$ be $$\rho_{\max}(x,y)=\max\limits_{1\leq k\leq n}\{|x_k-y_k|\}$$ Is $\rho_{max}$ a metric in $\mathbb R^2$?
\end{Exercise}

\begin{Exercise}[title=*]
	Draw the balls $B_1(0)$, i.e. the balls centered in the origin of radius 1, in the metric spaces $(\mathbb R^2,\rho_1)$, $(\mathbb R^2,\rho_2)$, $(\mathbb R^2,\rho_3)$, and $(\mathbb R^2,\rho_{max})$, where the distances $\rho_1, \rho_2,\rho_3$ and $\rho_{max}$ are defined as in the previous exercises.
\end{Exercise}

\begin{Exercise}[title=***]
	Show that $\rho_p$ tends to $\rho_{max}$ as $p\to\infty$ in any $\mathbb R^n$.
	
	[Hint: Try to bound $\rho_p$ both from below and above. Also consider splitting $p$ in $p-q+q$ for a fixed $q<p$.]
\end{Exercise}

\begin{Exercise}[title=*]
	Recall the definition of a bounded set in a metric space $(X,d)$. Show that a finite union of bounded sets is a bounded set. Is it true for a countable union of bounded sets? Why or why not? If not, can you show an example?
\end{Exercise}

\begin{Exercise}[title=*$\dagger$]
	 Let $X=[0,1]\cup[2,4)$, with $d(x,y)=|x-y|$.
	\begin{enumerate}[label={\alph*)}]
		\item $A=[2,4)$ is it open ? Is it closed ?
		\item Show that $B=[0,1]$ is both open and closed
		\item What about $X$?
	\end{enumerate}
\end{Exercise}

\begin{Exercise}[title=*$\dagger$]
	Decide, by using the definitions, whether the following sets are open or closed as subsets of $\mathbb R$
	\begin{enumerate}[label={\alph*)}]
		\item $\Q$
		\item $\N$
		\item $\R \backslash \{0\}$
		\item $\left\{\frac{1}{n}\bigst n\in\N^\ast\right\}$
		\item $\left\{\frac{1}{n^2}\bigst n\in\N^\ast\right\}$
	\end{enumerate}
\end{Exercise}


\begin{Exercise}[title=**]
	Prove that $\overline{A\cup B} = \overline A \cup \overline B$. Does the same result hold true when considering infinite unions of sets?
	
	
	[Hint: Consider $\Q$]
	
\end{Exercise}

\begin{Exercise}[title=**]
	Let
	\[\mathring{A}=\{x\;|\;x \text{ is an interior point of } A\},\]
	\[\bar{A}=\{x\;|\;x\in A \text{ or } x \text{ is a boundary point of } A\}.\]
	Show that $\mathring{A}$ is open and $\bar{A}$ is closed.
	
	[Hint: this is not trivial. We need to show that an interior point stays interior even when we remove the boundary, and that if we add the boundary then the boundary of this new set we have created is not "bigger"]
\end{Exercise}

\begin{Exercise}[title=*]
	\begin{enumerate}[label={\alph*)}]
	\item Show that arbitrary union of open sets is an open set.
	\item Show that finite intersection of open sets is open.
	\item Show with an example that countable intersection of open sets can be closed.
	\item Apply now De Morgan's laws to show that  arbitrary intersection of closed sets is closed and finite union of closed sets is closed.
	\item Show with an example that countable union of closed sets can be open.
	\end{enumerate}
\end{Exercise}

\end{document}

