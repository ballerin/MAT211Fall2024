% !TeX program = latexmk -pdf -pdflatex="pdflatex -synctex=1 -interaction=nonstopmode -shell-escape" -jobname=% -pretex="\newcommand{\version}{noanswer}" -usepretex % | latexmk -pdf -pdflatex="pdflatex -synctex=1 -interaction=nonstopmode -shell-escape" -jobname=%_solutions -pretex="\newcommand{\version}{}" -usepretex % | txs:///view-pdf "?am)_solutions.pdf"

\documentclass[11pt]{article}% autres choix : report, book

% setting a default value in case it is compiled without the magic comment
\unless\ifdefined\version
\def\version{noanswer}
\fi

\usepackage[utf8]{inputenc}
\usepackage[T1]{fontenc}
\usepackage[english]{babel}
\usepackage{textcomp}
\usepackage{amsmath,amssymb,amsthm}
\usepackage{pxfonts}
\usepackage[a4paper]{geometry}
\usepackage{graphicx}
\usepackage{float}
\usepackage{xcolor}
\usepackage{microtype}
\usepackage{enumitem}
\usepackage{hyperref}
\usepackage{pgfplots}
\usepackage[\version]{exercise}
\hypersetup{pdfstartview=XYZ}% zoom par défaut
\newtheoremstyle{exercice}%
%\usepackage{tikz}% Faire figure, graphique...
{\topsep}% espace avant
{\topsep}% espace après
{\upshape}% police du corps du théorème
{}% indentation (vide pour rien, \parindent)
{\bfseries}% police du titre du théorème
{}% ponctuation après le théorème
{ }% espace après le titre du théorème (\newline = saut de ligne)
{\thmname{#1}\thmnumber{ \textup{#2}}. ---\thmnote{ \textnormal{\itshape#3.}}}% spécification
% du titre du théorème

%Custom symbols
\newcommand{\R}{\mathbb{R}}
\newcommand{\C}{\mathbb{C}}
\newcommand{\Q}{\mathbb{Q}}
\newcommand{\N}{\mathbb{N}}
\newcommand{\e}{\mathrm{e}}
\newcommand{\eps}{\varepsilon}

\newcommand{\st}{\;|\;}
\newcommand{\bigst}{\;\Bigg|\;}

\newtheorem{theorem}{Theorem}

%Customization of the Answers package
\def\AnswerName{Solution of exercise}
%\newcommand{\AnswerHeader}{\medskip{\textbf{ Answer of \ExerciseName\ \ExerciseHeaderNB}\smallskip}}
\renewcommand{\ExerciseHeader}{%
	\par\noindent
	\textbf{\large \ExerciseName\ \ExerciseHeaderNB \ExerciseHeaderTitle\ExerciseHeaderOrigin}%
	\par\nopagebreak\medskip
}

\renewcommand{\AnswerHeader}{%
	\par\noindent
	\textbf{\large Solution of \ExerciseName\ \ExerciseHeaderNB \ExerciseHeaderTitle}%
	\par\nopagebreak\medskip
}
\setlength{\ExerciseSkipAfter}{1\baselineskip}
\setlength{\AnswerSkipAfter}{1\baselineskip}

\title{MAT 211 : Exercise Sheet 8}
\author{Francesco Ballerin}
\date{\color{gray}{\small{francesco.ballerin@uib.no}}}

\pagestyle{empty}


\begin{document}
\begin{minipage}[t]{\dimexpr \textwidth-6cm-\columnsep}
     \maketitle
\end{minipage}
\hfill\noindent\raisebox{-1.5\height}{\includegraphics[scale=0.1]{../UiBlogoMN.png}}

\vspace{50pt}



\begin{Exercise}[title=**] Show that the series 
$$
\sum_{n=2}^{\infty}\frac{1}{n(\log n)^p} 
$$
converges for $p>1$ and diverges for $p\leq 1$.
\end{Exercise}


\begin{Exercise}[title=**$\dagger$]
Let $(u_n)_{n\in\mathbb{N}}$ be a sequence in $\mathbb{R}$.
\begin{enumerate}
	\item Suppose that the sequence $(u_n)_{n\in\mathbb{N}}$ converges to $l$. Prove that $(u_{2n})_{n\in\mathbb{N}}$ and $(u_{2n+1})_{n\in\mathbb{N}}$ also converge to $l$. 
	\item Prove the inverse is not true, that is if $(u_{2n})_{n\in\mathbb{N}}$ and $(u_{2n+1})_{n\in\mathbb{N}}$ both converge then $(u_n)_{n\in\mathbb{N}}$ might not converge.
	\item Prove that however if $(u_{2n})_{n\in\mathbb{N}}$ and $(u_{2n+1})_{n\in\mathbb{N}}$ converge to the same limit $l$ then $(u_n)_{n\in\mathbb{N}}$ also converges to $l$.
\end{enumerate}
\end{Exercise}
\newpage

\begin{Exercise}[title=***]
Compute the radius of convergence of the following power series, where $z\in\mathbb{C}$ is a fixed complex number.
\begin{enumerate}
	\item $$\displaystyle\sum_{n\in\mathbb{N}\backslash\{0\}} \frac{z^n}{\sqrt{n}}$$
	\item $$\displaystyle\sum_{n\in\mathbb{N}}\frac{n!}{(2n)!}z^n$$
	\item $$\displaystyle\sum_{n\in\mathbb{N}\backslash\{0\}}\log(n)z^n$$
	\item $$\displaystyle\sum_{n\in\mathbb{N}}z^{n!}$$
	\item $$\displaystyle\sum_{n\in\mathbb{N}\backslash\{0\}}n^{\log(n)}z^n$$
	\item Let $a_n$ be the $n^{\text{th}}$ decimal of $\sqrt{2}$ $$\displaystyle\sum_{n\in\mathbb{N}}a_nz^n$$
\end{enumerate}
\end{Exercise}

\begin{Exercise}[title=***]
Write the following functions as power series, and compute the radius of convergence:
\begin{enumerate}
	\item $$f(x) = \frac{6}{1+7x^4}$$
	\item $$f(x) = \frac{x^3}{3-x^2}$$
	\item $$f(x) = \frac{3x^2}{5-2\sqrt[3]{x}}$$
	[Hint: rewrite the denominator to be in the form $1-p(x)$, where $p(x)$ depends on $x$, and recall the properties of geometric series]
\end{enumerate}
\end{Exercise}

\newpage

\begin{Exercise}[title=**$\dagger$]
For each of the following real-valued functions of a real variable give a well
written $\epsilon-\delta$ proof of the claim.
\begin{enumerate}
	\item $$\lim_{n\rightarrow2}(3x^2-2x + 1)=9$$
	\item $$\lim_{n\rightarrow-1}8x^2=8$$
	\item $$\lim_{n\rightarrow16}\sqrt{x}=4$$
	\item $$\lim_{n\rightarrow1}\frac{3}{x-2}=-3$$
	\item $$\lim_{n\rightarrow3}\frac{x+4}{2x-5}=7$$
\end{enumerate}
\end{Exercise}

\begin{Exercise}[title=**]
	 Let $$R(x)=\frac{a_nx^n+a_{n-1}x^{n-1}+\ldots+a_0}{b_mx^m+b_{m-1}x^{n-1}+\ldots+b_0},$$ where $a_n\neq 0$, $b_m\neq 0$ be a division between polynomials. Show that 
$$\lim\limits_{x\to\infty}R(x)=\left\{\begin{array}{lllc}\infty \quad & \text{if}\quad & n>m,\\
	\frac{a_n}{b_m} &\text{if}  & n=m,\\
	0&\text{if}  & n<m.
\end{array}\right.
$$
\end{Exercise}


\begin{Exercise}[title=**$\dagger$] Let $P(x)$ and $Q(x)$be polynomials on $x$ and $$P(a)=Q(a)=0.$$ What are the possible values of the limit $$\lim\limits_{x\to a}\frac{P(x)}{Q(x)}?$$ Explain your reasoning and provide examples for all the cases.
\end{Exercise}
\end{document}

