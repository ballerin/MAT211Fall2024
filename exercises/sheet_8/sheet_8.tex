% !TeX document-id = {13d6aedf-515c-41df-b0d9-c5edf94eb938}
% !TeX program = latexmk -pdf -pdflatex="pdflatex -synctex=1 -interaction=nonstopmode -shell-escape" -jobname=% -pretex="\newcommand{\version}{noanswer}" -usepretex % | latexmk -pdf -pdflatex="pdflatex -synctex=1 -interaction=nonstopmode -shell-escape" -jobname=%_solutions -pretex="\newcommand{\version}{}" -usepretex % | txs:///view-pdf "?am)_solutions.pdf"

\documentclass[11pt]{article}% autres choix : report, book

% setting a default value in case it is compiled without the magic comment
\unless\ifdefined\version
\def\version{noanswer}
\fi

\usepackage[utf8]{inputenc}
\usepackage[T1]{fontenc}
\usepackage[english]{babel}
\usepackage{textcomp}
\usepackage{amsmath,amssymb,amsthm}
\usepackage{pxfonts}
\usepackage[a4paper]{geometry}
\usepackage{graphicx}
\usepackage{float}
\usepackage{xcolor}
\usepackage{microtype}
\usepackage{enumitem}
\usepackage{hyperref}
\usepackage{pgfplots}
\usepackage[\version]{exercise}
\hypersetup{pdfstartview=XYZ}% zoom par défaut
\newtheoremstyle{exercice}%
%\usepackage{tikz}% Faire figure, graphique...
{\topsep}% espace avant
{\topsep}% espace après
{\upshape}% police du corps du théorème
{}% indentation (vide pour rien, \parindent)
{\bfseries}% police du titre du théorème
{}% ponctuation après le théorème
{ }% espace après le titre du théorème (\newline = saut de ligne)
{\thmname{#1}\thmnumber{ \textup{#2}}. ---\thmnote{ \textnormal{\itshape#3.}}}% spécification
% du titre du théorème

%Custom symbols
\newcommand{\R}{\mathbb{R}}
\newcommand{\C}{\mathbb{C}}
\newcommand{\Q}{\mathbb{Q}}
\newcommand{\N}{\mathbb{N}}
\newcommand{\e}{\mathrm{e}}
\newcommand{\eps}{\varepsilon}

\newcommand{\st}{\;|\;}
\newcommand{\bigst}{\;\Bigg|\;}

\newtheorem{theorem}{Theorem}

%Customization of the Answers package
\def\AnswerName{Solution of exercise}
%\newcommand{\AnswerHeader}{\medskip{\textbf{ Answer of \ExerciseName\ \ExerciseHeaderNB}\smallskip}}
\renewcommand{\ExerciseHeader}{%
	\par\noindent
	\textbf{\large \ExerciseName\ \ExerciseHeaderNB \ExerciseHeaderTitle\ExerciseHeaderOrigin}%
	\par\nopagebreak\medskip
}

\renewcommand{\AnswerHeader}{%
	\par\noindent
	\textbf{\large Solution of \ExerciseName\ \ExerciseHeaderNB \ExerciseHeaderTitle}%
	\par\nopagebreak\medskip
}
\setlength{\ExerciseSkipAfter}{1\baselineskip}
\setlength{\AnswerSkipAfter}{1\baselineskip}

\title{MAT 211 : Exercise Sheet 8}
\author{Francesco Ballerin}
\date{\color{gray}{\small{francesco.ballerin@uib.no}}}

\pagestyle{empty}


\begin{document}
\begin{minipage}[t]{\dimexpr \textwidth-6cm-\columnsep}
     \maketitle
\end{minipage}
\hfill\noindent\raisebox{-1.5\height}{\includegraphics[scale=0.1]{../UiBlogoMN.png}}

\vspace{50pt}

\begin{Exercise}[title={*}]
	Let $f,g:\R\to\R$. Find an example, or prove the impossibility of finding one, for the following requests:
	\begin{enumerate}
		\item $f$ and $g$ not differentiable at zero so that $fg$ differentiable at zero.
		\item $f$ not differentiable at zero and $g$  differentiable at 0 so that $fg$ differentiable at zero.
		\item $f$ not differentiable at zero and $g$ differentiable at 0 so that $f+g$ differentiable at zero.
		\item $f$ differentiable at zero but not at any other point (or more in general a function that is only differentiable at a point)
	\end{enumerate}
\end{Exercise}

\begin{Exercise}[title={*}]
	Let $$\displaystyle I_n=\int_{1}^2 e^{-nx^2} dx.$$ Consider computing directly $I_n$ (hard task). It is however not difficult to compute the quantity $$\lim_{n\to\infty} I_n.$$
	What conclusion can you draw?
\end{Exercise}

\begin{Exercise}[title={*}]
	Argue that every function of an equicontinuous family is uniformly continuous. Construct a family of uniformly continuous functions that is not an equicontinuous family.
\end{Exercise}

\begin{Exercise}[title={*}]
	Suppose $f\geq0$ is a continuous function on $[a,b]$. Show that $\int_a^b f(x)dx=0$ implies that $\forall x\in[a,b]\;\;f(x)=0$.
	
	\bigskip
	[Hint: proceed by contradiction]
\end{Exercise}


\bigskip

\begin{Exercise}[title={**}]
	Let $f$ be a continuous function on $[0,1]$ such that $\forall n\in \mathbb{N}$
	$$\int_0^1 f(x) x^n dx =0$$
	Show that $f(x)=0$ on $[0,1]$.
	
	\bigskip
	
	[Hint: This means that the integral of $f$ multiblied by any polynomial is zero. Use the Stone-Weierstrass theorem to show that $\int_0^1 f^2(x)dx=0$]
\end{Exercise}

\begin{Exercise}[title={*}]
	Assume that $(V_1,\lVert\cdot\rVert_1)$, $(V_2,\lVert\cdot\rVert_2)$, ..., $(V_n,\lVert\cdot\rVert_n)$ are vector spaces over $\mathbb K$. Define $\lVert\cdot\rVert$ on the product space $V = V_1\times V_2\times\cdots\times V_n$ as
	\[\lVert\textbf{v}\rVert = \lVert\textbf{v}_1\rVert_1+\lVert\textbf{v}_2\rVert_2+\dots + \lVert\textbf{v}_n\rVert_n\] 
	where $\textbf{v} = (\textbf{v}_1, \textbf{v}_2, \dots, \textbf{v}_n)$.
	\begin{enumerate}
		\item Show that $\lVert\cdot\rVert$ is a norm.
		\item Show that if the spaces $(V_1,\lVert\cdot\rVert_1)$, $(V_2,\lVert\cdot\rVert_2)$, ..., $(V_n,\lVert\cdot\rVert_n)$ are complete then also $(V,\lVert\cdot\rVert)$ is complete.
	\end{enumerate}
	
\end{Exercise}

\begin{Exercise}[title={**}]
	Show that the discrete distance on a vector space $V\neq {0}$ cannot be generated by a norm, i.e. there is no norm $\lVert\cdot\rVert$ such that $d(x,y) = \lVert x-y \rVert$ where $d$ is the discrete distance.
\end{Exercise}

\begin{Exercise}[title={**}]
	Show that a vector space $V\neq {0}$ is complete if and only if the unit sphere \[S:=\{v\in V \;|\; \lVert v\rVert =1\}\]
	is complete.
\end{Exercise}



\end{document}

