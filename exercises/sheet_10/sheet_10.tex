% !TeX program = latexmk -pdf -pdflatex="pdflatex -synctex=1 -interaction=nonstopmode -shell-escape" -jobname=% -pretex="\newcommand{\version}{noanswer}" -usepretex % | latexmk -pdf -pdflatex="pdflatex -synctex=1 -interaction=nonstopmode -shell-escape" -jobname=%_solutions -pretex="\newcommand{\version}{}" -usepretex % | txs:///view-pdf "?am)_solutions.pdf"

\documentclass[11pt]{article}% autres choix : report, book

% setting a default value in case it is compiled without the magic comment
\unless\ifdefined\version
\def\version{noanswer}
\fi

\usepackage[utf8]{inputenc}
\usepackage[T1]{fontenc}
\usepackage[english]{babel}
\usepackage{textcomp}
\usepackage{amsmath,amssymb,amsthm}
\usepackage{pxfonts}
\usepackage[a4paper]{geometry}
\usepackage{graphicx}
\usepackage{float}
\usepackage{xcolor}
\usepackage{microtype}
\usepackage{enumitem}
\usepackage{hyperref}
\usepackage{pgfplots}
\usepackage[\version]{exercise}
\hypersetup{pdfstartview=XYZ}% zoom par défaut
\newtheoremstyle{exercice}%
%\usepackage{tikz}% Faire figure, graphique...
{\topsep}% espace avant
{\topsep}% espace après
{\upshape}% police du corps du théorème
{}% indentation (vide pour rien, \parindent)
{\bfseries}% police du titre du théorème
{}% ponctuation après le théorème
{ }% espace après le titre du théorème (\newline = saut de ligne)
{\thmname{#1}\thmnumber{ \textup{#2}}. ---\thmnote{ \textnormal{\itshape#3.}}}% spécification
% du titre du théorème

%Custom symbols
\newcommand{\R}{\mathbb{R}}
\newcommand{\C}{\mathbb{C}}
\newcommand{\Q}{\mathbb{Q}}
\newcommand{\N}{\mathbb{N}}
\newcommand{\e}{\mathrm{e}}
\newcommand{\eps}{\varepsilon}

\newcommand{\st}{\;|\;}
\newcommand{\bigst}{\;\Bigg|\;}

\newtheorem{theorem}{Theorem}

%Customization of the Answers package
\def\AnswerName{Solution of exercise}
%\newcommand{\AnswerHeader}{\medskip{\textbf{ Answer of \ExerciseName\ \ExerciseHeaderNB}\smallskip}}
\renewcommand{\ExerciseHeader}{%
	\par\noindent
	\textbf{\large \ExerciseName\ \ExerciseHeaderNB \ExerciseHeaderTitle\ExerciseHeaderOrigin}%
	\par\nopagebreak\medskip
}

\renewcommand{\AnswerHeader}{%
	\par\noindent
	\textbf{\large Solution of \ExerciseName\ \ExerciseHeaderNB \ExerciseHeaderTitle}%
	\par\nopagebreak\medskip
}
\setlength{\ExerciseSkipAfter}{1\baselineskip}
\setlength{\AnswerSkipAfter}{1\baselineskip}

\title{MAT 211 : Exercise Sheet 10}
\author{Francesco Ballerin}
\date{\color{gray}{\small{francesco.ballerin@uib.no}}}

\pagestyle{empty}


\begin{document}
\begin{minipage}[t]{\dimexpr \textwidth-6cm-\columnsep}
     \maketitle
\end{minipage}
\hfill\noindent\raisebox{-1.5\height}{\includegraphics[scale=0.1]{../UiBlogoMN.png}}

\vspace{50pt}


\bigskip


\begin{Exercise}[title=**$\dagger$]
Consider the function 
\[f(x) = 
\begin{cases} 
	\frac{x^2y}{x^4+y^2} & \text{if } (x,y)\neq(0,0) \\
	0 & \text{if } (x,y)=(0,0)
\end{cases}\]
\begin{enumerate}
	\item Show that the directional derivatives $\frac{\partial}{\partial x}$ and $\frac{\partial}{\partial y}$ exist and are zero.
	\item Show that the directional derivatives along any straight line by the origin exist and are all zero.
	\item Show that the function is not continuous by computing the limit along the curve $r(t)=(t,t^2)$.
	\item How is it possible that the function is differentiable in all directions but fails to be continuous?
\end{enumerate}
\end{Exercise}

\begin{Exercise}[title=*$\dagger$]
	Assume that $\textbf{F, G}:X\to Y$ are differentiable at $\textbf{a}\in X$. Show that for any constant $\alpha, \beta \in\mathbb K$ (scalars) the function $\alpha F+\beta G$ is differentiable at $\textbf{a}$ with derivative $\alpha F'+\beta G'$.
\end{Exercise}

\begin{Exercise}[title=**$\dagger$]
	Let $X=Y=C([0,1], \mathbb R)$ with the usual supremum norm. Consider the map $\textbf{F}:X\to Y$ defined by
	\[\textbf{F}(y)(x) = \int_0^x y(s)ds.\]
	\begin{enumerate}
		\item Recall that all functions in $X=Y$ are uniformly continuous.
		\item Compute the directional derivatives for $\textbf{F}$ to find a natural candidate for the Fréchet derivative.
		\item Show that this candidate is linear and bounded
		\item Compute $\sigma(r)$ and show that the condition for the Fréchet derivative holds
	\end{enumerate}
\end{Exercise}

\begin{Exercise}[title=*]
	Compute the Jacobian matrices in Exercises 6.2.1 and 6.2.2 of Lindtstrøm. Show that the functions are differentiable and compute the required directional derivatives.
\end{Exercise}

\begin{Exercise}[title=*]
	Let $X,Y$ be two normed vector spaces and let $O\subset X$ be open and convex.
	\begin{enumerate}
		\item Show that if $F:O\to Y$ is differentiable with $F'(x)=0$ then $F$ is a constant.
		\item Show that if $G,H:O\to Y$ are differentiable with $G'(x)=H'(x)$ then $G$ and $H$ differ by a constant (i.e. $H(x)=G(x)+C$ where $C\in Y$).
	\end{enumerate}
	[Hint: use the Mean Value Theorem]
\end{Exercise}

\begin{Exercise}[title=**]
	Revise Taylor expansions and write the corresponding theorems for functions $\R^n\to \R^m$.
	
	[Hint: pages 198-199]
\end{Exercise}

\end{document}

