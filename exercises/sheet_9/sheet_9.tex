% !TeX document-id = {06aa16ad-6e89-4da6-a747-c5cbe5f38bcf}
% !TeX program = latexmk -pdf -pdflatex="pdflatex -synctex=1 -interaction=nonstopmode -shell-escape" -jobname=% -pretex="\newcommand{\version}{noanswer}" -usepretex % | latexmk -pdf -pdflatex="pdflatex -synctex=1 -interaction=nonstopmode -shell-escape" -jobname=%_solutions -pretex="\newcommand{\version}{}" -usepretex % | txs:///view-pdf "?am)_solutions.pdf"

\documentclass[11pt]{article}% autres choix : report, book

% setting a default value in case it is compiled without the magic comment
\unless\ifdefined\version
\def\version{noanswer}
\fi

\usepackage[utf8]{inputenc}
\usepackage[T1]{fontenc}
\usepackage[english]{babel}
\usepackage{textcomp}
\usepackage{amsmath,amssymb,amsthm}
\usepackage{pxfonts}
\usepackage[a4paper]{geometry}
\usepackage{graphicx}
\usepackage{float}
\usepackage{xcolor}
\usepackage{microtype}
\usepackage{enumitem}
\usepackage{hyperref}
\usepackage{pgfplots}
\usepackage[\version]{exercise}
\hypersetup{pdfstartview=XYZ}% zoom par défaut
\newtheoremstyle{exercice}%
%\usepackage{tikz}% Faire figure, graphique...
{\topsep}% espace avant
{\topsep}% espace après
{\upshape}% police du corps du théorème
{}% indentation (vide pour rien, \parindent)
{\bfseries}% police du titre du théorème
{}% ponctuation après le théorème
{ }% espace après le titre du théorème (\newline = saut de ligne)
{\thmname{#1}\thmnumber{ \textup{#2}}. ---\thmnote{ \textnormal{\itshape#3.}}}% spécification
% du titre du théorème

%Custom symbols
\newcommand{\R}{\mathbb{R}}
\newcommand{\C}{\mathbb{C}}
\newcommand{\Q}{\mathbb{Q}}
\newcommand{\N}{\mathbb{N}}
\newcommand{\e}{\mathrm{e}}
\newcommand{\eps}{\varepsilon}

\newcommand{\st}{\;|\;}
\newcommand{\bigst}{\;\Bigg|\;}

\newtheorem{theorem}{Theorem}

%Customization of the Answers package
\def\AnswerName{Solution of exercise}
%\newcommand{\AnswerHeader}{\medskip{\textbf{ Answer of \ExerciseName\ \ExerciseHeaderNB}\smallskip}}
\renewcommand{\ExerciseHeader}{%
	\par\noindent
	\textbf{\large \ExerciseName\ \ExerciseHeaderNB \ExerciseHeaderTitle\ExerciseHeaderOrigin}%
	\par\nopagebreak\medskip
}

\renewcommand{\AnswerHeader}{%
	\par\noindent
	\textbf{\large Solution of \ExerciseName\ \ExerciseHeaderNB \ExerciseHeaderTitle}%
	\par\nopagebreak\medskip
}
\setlength{\ExerciseSkipAfter}{1\baselineskip}
\setlength{\AnswerSkipAfter}{1\baselineskip}

\title{MAT 211 : Exercise Sheet 9}
\author{Francesco Ballerin}
\date{\color{gray}{\small{francesco.ballerin@uib.no}}}

\pagestyle{empty}


\begin{document}
\begin{minipage}[t]{\dimexpr \textwidth-6cm-\columnsep}
     \maketitle
\end{minipage}
\hfill\noindent\raisebox{-1.5\height}{\includegraphics[scale=0.1]{../UiBlogoMN.png}}

\vspace{50pt}

\begin{Exercise}[title={**$\dagger$}]
	Revise theorem 3.55 of Rudin. Let $\sum_{n=0}^{\infty}a_n$ be a series of complex numbers which converges absolutely. Show that every rearrangement also converges and it does converge to the same sum. 
\end{Exercise}

\begin{Exercise}[title=*]
	Let $(X,d)$ be a metric space and $E\subset X$. Let $f:E\to\mathbb{R}$ be a function. We say that $f(t)\to A$ as $t\to x$, where $t,x\in\overline\R$ are numbers of the extended real line ($\R\cup\{-\infty,+\infty\}$), if for every neighborhood $U$ of $A$ there is a neighborhood $V$ of $x$ such that $V\cap E$ is not empty and such that $f(t)\in U$ for all $t\in V\cap E$. Define any interval $(c,+\infty)$ for $c\in\R$ to be a neighborhood of $+\infty$, and  $(-\infty,c)$ to be a neighborhood of $-\infty$.
	
	Check that this new definition of limit of a function corresponds to the usual one when $A,x\in \R$.
\end{Exercise}

\begin{Exercise}[title=**]
	Revise Theorem 4.20 of Rudin, i.e. that in a noncompact subset of $\R$, either bounded or unbounded, there exists:
	\begin{enumerate}
		\item a continuous function that is not bounded
		\item a continuous function that is bounded but has no maximum
		\item a continuous function that is not uniformly continuous when $E$ is bounded 
	\end{enumerate}
	
	In addition show that there are choices for $E$ unbounded such that all continuous functions are also uniformly continuous.
\end{Exercise}

\begin{Exercise}[title=**]
	Revise Theorem 4.22 of Rudin, i.e. that $f(E)$ is connected whenever $f:E\to Y$, for $E\subset X$ a connected set, is a continuous map between metric spaces.
\end{Exercise}

\begin{Exercise}[title=*]
	Can a monotonic function admit a discontinuity of the second kind? Provide an example or argue that it is not possible.
\end{Exercise}

\begin{Exercise}[title=**]
	Revise Theorem 4.30 of Rudin, i.e. that a monotonic real function defined on $(a,b)$ has at most countably many discontinuities of the first kind. Can you come up with a function that is monotonic in $(a,b)$ and has a countably infinite number of discontinuities?
\end{Exercise}

\begin{Exercise}[title=**$\dagger$]
For each of the following real-valued functions of a real variable give a well written $\epsilon-\delta$ proof of the claim
\begin{enumerate}
	\item $$\lim_{x\rightarrow 1} x = 1$$
	\item $$\lim_{x\rightarrow 2} x^2+3 = 7$$
	\item $$\lim_{x\rightarrow 1} \frac{x^2-1}{x-1} = 0$$
	\item $$\lim_{x\rightarrow 0} e^x = 1$$
\end{enumerate}

\end{Exercise}




\end{document}

