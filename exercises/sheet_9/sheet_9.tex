% !TeX document-id = {06aa16ad-6e89-4da6-a747-c5cbe5f38bcf}
% !TeX program = latexmk -pdf -pdflatex="pdflatex -synctex=1 -interaction=nonstopmode -shell-escape" -jobname=% -pretex="\newcommand{\version}{noanswer}" -usepretex % | latexmk -pdf -pdflatex="pdflatex -synctex=1 -interaction=nonstopmode -shell-escape" -jobname=%_solutions -pretex="\newcommand{\version}{}" -usepretex % | txs:///view-pdf "?am)_solutions.pdf"

\documentclass[11pt]{article}% autres choix : report, book

% setting a default value in case it is compiled without the magic comment
\unless\ifdefined\version
\def\version{noanswer}
\fi

\usepackage[utf8]{inputenc}
\usepackage[T1]{fontenc}
\usepackage[english]{babel}
\usepackage{textcomp}
\usepackage{amsmath,amssymb,amsthm}
\usepackage{pxfonts}
\usepackage[a4paper]{geometry}
\usepackage{graphicx}
\usepackage{float}
\usepackage{xcolor}
\usepackage{microtype}
\usepackage{enumitem}
\usepackage{hyperref}
\usepackage{pgfplots}
\usepackage[\version]{exercise}
\hypersetup{pdfstartview=XYZ}% zoom par défaut
\newtheoremstyle{exercice}%
%\usepackage{tikz}% Faire figure, graphique...
{\topsep}% espace avant
{\topsep}% espace après
{\upshape}% police du corps du théorème
{}% indentation (vide pour rien, \parindent)
{\bfseries}% police du titre du théorème
{}% ponctuation après le théorème
{ }% espace après le titre du théorème (\newline = saut de ligne)
{\thmname{#1}\thmnumber{ \textup{#2}}. ---\thmnote{ \textnormal{\itshape#3.}}}% spécification
% du titre du théorème

%Custom symbols
\newcommand{\R}{\mathbb{R}}
\newcommand{\C}{\mathbb{C}}
\newcommand{\Q}{\mathbb{Q}}
\newcommand{\N}{\mathbb{N}}
\newcommand{\e}{\mathrm{e}}
\newcommand{\eps}{\varepsilon}

\newcommand{\st}{\;|\;}
\newcommand{\bigst}{\;\Bigg|\;}

\newtheorem{theorem}{Theorem}

%Customization of the Answers package
\def\AnswerName{Solution of exercise}
%\newcommand{\AnswerHeader}{\medskip{\textbf{ Answer of \ExerciseName\ \ExerciseHeaderNB}\smallskip}}
\renewcommand{\ExerciseHeader}{%
	\par\noindent
	\textbf{\large \ExerciseName\ \ExerciseHeaderNB \ExerciseHeaderTitle\ExerciseHeaderOrigin}%
	\par\nopagebreak\medskip
}

\renewcommand{\AnswerHeader}{%
	\par\noindent
	\textbf{\large Solution of \ExerciseName\ \ExerciseHeaderNB \ExerciseHeaderTitle}%
	\par\nopagebreak\medskip
}
\setlength{\ExerciseSkipAfter}{1\baselineskip}
\setlength{\AnswerSkipAfter}{1\baselineskip}

\title{MAT 211 : Exercise Sheet 9}
\author{Francesco Ballerin}
\date{\color{gray}{\small{francesco.ballerin@uib.no}}}

\pagestyle{empty}


\begin{document}
\begin{minipage}[t]{\dimexpr \textwidth-6cm-\columnsep}
     \maketitle
\end{minipage}
\hfill\noindent\raisebox{-1.5\height}{\includegraphics[scale=0.1]{../UiBlogoMN.png}}

\vspace{50pt}

\begin{Exercise}[title={**$\dagger$}]
	Let $c_0$ be the set of all the sequences $\textbf{x}=(x_n)_n$ of real numbers such that $x_n\to0$.
	\begin{enumerate}
		\item Check that $c_0$ is a vector space.
		\item Show that \[\lVert \textbf{x} \rVert = \sup\{|x_k|\}\] is a norm on $c_0$.
		\item Prove that the set $\{\textbf{e}_k\}$, where $\textbf{e}_k = (0,0,\dots,0,1,0,\dots)$, is a basis for $c_0$.
	\end{enumerate}
\end{Exercise}

\begin{Exercise}[title={**$\dagger$}]
	Let $l_1$ be the set of all the sequences $\textbf{x}=(x_n)_n$ of real numbers such that $\sum_{n=0}^\infty |x_n|$ converges.
	\begin{enumerate}
		\item Show that $l_1$ is a vector space.
		\item Show that \[\lVert \textbf{x} \rVert = \sum_{n=0}^\infty |x_n|\] is a norm on $l_1$.
		\item Prove that the set $\{\textbf{e}_k\}$, where $\textbf{e}_k = (0,0,\dots,0,1,0,\dots)$, is a basis for $l_1$.
		\item Show that $l_1$ is complete.
	\end{enumerate}
\end{Exercise}

\begin{Exercise}[title={**$\dagger$}]
	Assume that $(u_n)_n$ and $(v_n)_n$ are two converging sequences in an inner product space, converging to $u$ and $v$ respectively. Show that $(\langle u_n, v_n\rangle)_n$ converges to $\langle u, v\rangle$
\end{Exercise}

\begin{Exercise}[title={**}]
	Show that whenever the norm is defines from an inner product, as $\lVert u \rVert = \langle u,u\rangle^\frac{1}{2}$ the following equality holds for any two vectors $u,v$:
	\[\lVert u+v \rVert^2+\lVert u-v \rVert^2 = 2(\lVert u \rVert^2+\lVert v \rVert^2)\]
\end{Exercise}

\begin{Exercise}[title={*}]
	Let $S$ be a nonempty set in an inner product space $V$ and define
	\[S^\perp = \{u\in V \;\;|\; \langle u,s\rangle=0\;\forall s\in S\}.\]
	Show that $S^\perp$ is closed.
	
	[Hint: take a convergent sequence and show that the limit belong to the set]
\end{Exercise}






\end{document}

