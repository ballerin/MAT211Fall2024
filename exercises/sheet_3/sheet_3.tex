% !TeX program = latexmk -pdf -pdflatex="pdflatex -synctex=1 -interaction=nonstopmode -shell-escape" -jobname=% -pretex="\newcommand{\version}{noanswer}" -usepretex % | latexmk -pdf -pdflatex="pdflatex -synctex=1 -interaction=nonstopmode -shell-escape" -jobname=%_solutions -pretex="\newcommand{\version}{}" -usepretex % | txs:///view-pdf "?am)_solutions.pdf"

\documentclass[11pt]{article}% autres choix : report, book

% setting a default value in case it is compiled without the magic comment
\unless\ifdefined\version
\def\version{noanswer}
\fi

\usepackage[utf8]{inputenc}
\usepackage[T1]{fontenc}
\usepackage[english]{babel}
\usepackage{textcomp}
\usepackage{amsmath,amssymb,amsthm}
\usepackage{pxfonts}
\usepackage[a4paper]{geometry}
\usepackage{graphicx}
\usepackage{float}
\usepackage{xcolor}
\usepackage{microtype}
\usepackage{enumitem}
\usepackage{hyperref}
\usepackage{pgfplots}
\usepackage[\version]{exercise}
\hypersetup{pdfstartview=XYZ}% zoom par défaut
\newtheoremstyle{exercice}%
%\usepackage{tikz}% Faire figure, graphique...
{\topsep}% espace avant
{\topsep}% espace après
{\upshape}% police du corps du théorème
{}% indentation (vide pour rien, \parindent)
{\bfseries}% police du titre du théorème
{}% ponctuation après le théorème
{ }% espace après le titre du théorème (\newline = saut de ligne)
{\thmname{#1}\thmnumber{ \textup{#2}}. ---\thmnote{ \textnormal{\itshape#3.}}}% spécification
% du titre du théorème

%Custom symbols
\newcommand{\R}{\mathbb{R}}
\newcommand{\C}{\mathbb{C}}
\newcommand{\Q}{\mathbb{Q}}
\newcommand{\N}{\mathbb{N}}
\newcommand{\e}{\mathrm{e}}
\newcommand{\eps}{\varepsilon}

\newcommand{\st}{\;|\;}
\newcommand{\bigst}{\;\Bigg|\;}

\newtheorem{theorem}{Theorem}

%Customization of the Answers package
\def\AnswerName{Solution of exercise}
%\newcommand{\AnswerHeader}{\medskip{\textbf{ Answer of \ExerciseName\ \ExerciseHeaderNB}\smallskip}}
\renewcommand{\ExerciseHeader}{%
	\par\noindent
	\textbf{\large \ExerciseName\ \ExerciseHeaderNB \ExerciseHeaderTitle\ExerciseHeaderOrigin}%
	\par\nopagebreak\medskip
}

\renewcommand{\AnswerHeader}{%
	\par\noindent
	\textbf{\large Solution of \ExerciseName\ \ExerciseHeaderNB \ExerciseHeaderTitle}%
	\par\nopagebreak\medskip
}
\setlength{\ExerciseSkipAfter}{1\baselineskip}
\setlength{\AnswerSkipAfter}{1\baselineskip}

\title{MAT 211 : Exercise Sheet 3}
\author{Francesco Ballerin}
\date{\color{gray}{\small{francesco.ballerin@uib.no}}}

\pagestyle{empty}


\begin{document}
\begin{minipage}[t]{\dimexpr \textwidth-6cm-\columnsep}
     \maketitle
\end{minipage}
\hfill\noindent\raisebox{-1.5\height}{\includegraphics[scale=0.1]{../UiBlogoMN.png}}

\vspace{50pt}

\begin{Exercise}[title=**$\dagger$]
	Show that for a sequence $(x_n)_n$ the following are equivalent:
	\begin{enumerate}
		\item the sequence $(x_n)_n$ converges
		\item all subsequences of $(x_n)_n$ converge
	\end{enumerate}
	
	What if we replace the second condition by with "all proper subsequences of $(x_n)_n$ converge", where a proper subsequence is a subsequence of $(x_n)_n$ which is not $(x_n)_n$ itself. What if we replace the second condition by "at least one subsequence of $(x_n)_n$ converges"?
\end{Exercise}

\begin{Exercise}[title=**$\dagger$]
	Show that in $\mathbb{R}$ a sequence converges if and only if it has the Cauchy property.
\end{Exercise}

\begin{Exercise}[title=**]
	Show that every sequence contains a monotone subsequence. Explain why this gives us a new proof for Bolzano-Weierstrass in the case of a bounded sequence.
\end{Exercise}

\begin{Exercise}[title=**]
	Prove the following statements for $n\in \mathbb{N}$ and $x,p\in\mathbb{R}$:
	\begin{enumerate}
		\item[i)]{If $p>0$, then $$\lim\limits_{n\to\infty}\frac{1}{n^p}=0$$}
		\item[ii)]{If $p>0$, then $$\lim\limits_{n\to\infty}\sqrt[n]{p}=1$$}
		\item[iii)]{If $|x|<1$, then $$\lim\limits_{n\to\infty}x^n=0$$}
	\end{enumerate}
\end{Exercise}

\begin{Exercise}[title=**$\dagger$]
	Let $a,b\in\R^n$ so that $a=(a_1,\ldots, a_n)$ and $b=(b_1,\ldots, b_n)$, where $a_k$ and $b_k$ are real numbers. Revise the proof of the Cauchy-Bunyakovsky-Schwarz inequality 
	$$
	\Big(\sum_{k=1}^{n}a_kb_k\Big)^2\leq \Big(\sum_{k=1}^{n}a_k^2\Big)\cdot\Big(\sum_{k=1}^{n}b_k^2\Big).
	$$
	Under what conditions is it actually an equality rather than an inequality?
	\bigskip
\end{Exercise}

\begin{Exercise}[title=**]
	Let $a=(a_1,\ldots, a_n)$ and $b=(b_1,\ldots, b_n)$, where $a_k$ and $b_k$ are real numbers. Revise the proof of the Cauchy-Bunyakovsky-Schwarz inequality 
	$$
	\Big(\sum_{k=1}^{n}a_kb_k\Big)^2\leq \Big(\sum_{k=1}^{n}a_k^2\Big)\cdot\Big(\sum_{k=1}^{n}b_k^2\Big).
	$$
	Under what conditions is it actually an equality rather than an inequality?
	\bigskip
	
	\noindent {\bf 7)**} Prove the Minkowskii inequality: $$\Big(\sum_{k=1}^{n}(a_k+b_k)^2\Big)^{1/2}\leq\Big(\sum_{k=1}^{n}a_k^2\Big)^{1/2}+\Big(\sum_{k=1}^{n}b_k^2\Big)^{1/2}
	$$ 
	by making use the Cauchy-Bunyakovsky-Schwarz inequality. This gives us the triangle inequality $\|\mathbf a+\mathbf b\|\leq\|\mathbf a\|+\|\mathbf b\|$ for vectors in $\mathbb R^n$, where $\mathbf a=(a_1,\ldots,a_n)$, $\mathbf b=(b_1,\ldots,b_n)$ and $\|\mathbf a\|=\Big(\sum_{k=1}^{n}a_k^2\Big)^{1/2}$.
	\bigskip
\end{Exercise}

\begin{Exercise} [title=*$\dagger$]
	Using the definition show that the following spaces are metric spaces, i.e. show that the three axioms of a metric space hold:
	\begin{enumerate}
		\item $(\R,d)$ where $d(x,y)=|x-y|$ and $|\cdot|$ is the absolute value.
		\item $(\mathbb Z,d)$ where $d(x,y)=|x-y|$ and $|\cdot|$ is the absolute value.
		\item $(\R^n,d)$ for $n\in\N$, $n\geq 2$, where $d(x,y)=\lVert x-y\rVert$ and $\lVert \cdot\rVert$ is the norm of a vector.
		\item $(\R,d)$ where $d(x,y)=1$ if $x=y$ and $d(x,y)=0$ if $x\neq y$.
	\end{enumerate}
\end{Exercise}

\end{document}

