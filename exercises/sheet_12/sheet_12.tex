% \textsc{}!TeX program = latexmk -pdf -pdflatex="pdflatex -synctex=1 -interaction=nonstopmode -shell-escape" -jobname=% -pretex="\newcommand{\version}{noanswer}" -usepretex % | latexmk -pdf -pdflatex="pdflatex -synctex=1 -interaction=nonstopmode -shell-escape" -jobname=%_solutions -pretex="\newcommand{\version}{}" -usepretex % | txs:///view-pdf "?am)_solutions.pdf"

\documentclass[11pt]{article}% autres choix : report, book

% setting a default value in case it is compiled without the magic comment
\unless\ifdefined\version
\def\version{noanswer}
\fi

\usepackage[utf8]{inputenc}
\usepackage[T1]{fontenc}
\usepackage[english]{babel}
\usepackage{textcomp}
\usepackage{amsmath,amssymb,amsthm}
\usepackage{pxfonts}
\usepackage[a4paper]{geometry}
\usepackage{graphicx}
\usepackage{float}
\usepackage{xcolor}
\usepackage{microtype}
\usepackage{enumitem}
\usepackage{hyperref}
\usepackage{pgfplots}
\usepackage[\version]{exercise}
\hypersetup{pdfstartview=XYZ}% zoom par défaut
\newtheoremstyle{exercice}%
%\usepackage{tikz}% Faire figure, graphique...
{\topsep}% espace avant
{\topsep}% espace après
{\upshape}% police du corps du théorème
{}% indentation (vide pour rien, \parindent)
{\bfseries}% police du titre du théorème
{}% ponctuation après le théorème
{ }% espace après le titre du théorème (\newline = saut de ligne)
{\thmname{#1}\thmnumber{ \textup{#2}}. ---\thmnote{ \textnormal{\itshape#3.}}}% spécification
% du titre du théorème

%Custom symbols
\newcommand{\R}{\mathbb{R}}
\newcommand{\C}{\mathbb{C}}
\newcommand{\Q}{\mathbb{Q}}
\newcommand{\N}{\mathbb{N}}
\newcommand{\e}{\mathrm{e}}
\newcommand{\eps}{\varepsilon}

\newcommand{\st}{\;|\;}
\newcommand{\bigst}{\;\Bigg|\;}

\newtheorem{theorem}{Theorem}

%Customization of the Answers package
\def\AnswerName{Solution of exercise}
%\newcommand{\AnswerHeader}{\medskip{\textbf{ Answer of \ExerciseName\ \ExerciseHeaderNB}\smallskip}}
\renewcommand{\ExerciseHeader}{%
	\par\noindent
	\textbf{\large \ExerciseName\ \ExerciseHeaderNB \ExerciseHeaderTitle\ExerciseHeaderOrigin}%
	\par\nopagebreak\medskip
}

\renewcommand{\AnswerHeader}{%
	\par\noindent
	\textbf{\large Solution of \ExerciseName\ \ExerciseHeaderNB \ExerciseHeaderTitle}%
	\par\nopagebreak\medskip
}
\setlength{\ExerciseSkipAfter}{1\baselineskip}
\setlength{\AnswerSkipAfter}{1\baselineskip}

\title{MAT 211 : Exercise Sheet 12}
\author{Francesco Ballerin}
\date{\color{gray}{\small{francesco.ballerin@uib.no}}}

\pagestyle{empty}


\begin{document}
\begin{minipage}[t]{\dimexpr \textwidth-6cm-\columnsep}
     \maketitle
\end{minipage}
\hfill\noindent\raisebox{-1.5\height}{\includegraphics[scale=0.1]{../UiBlogoMN.png}}

\vspace{50pt}

\begin{Exercise}[title={*}]
	Let us show that the converse of Theorem 7.10 of Rudin (where 7.10 is also known as the Weierstrass M-test) does not hold. It is clear that if $\sum f_n$ converges uniformly and $(M_n)_n$ defined as we please s.t. $\lvert f_n(x)\rvert \leq M_n$ then $\sum M_n$ does not need to converge (if we take $M_n$ to be a constant w.r.t $n$ then the sum clearly diverges to $+\infty$). Show that this could happen also when $M_n:= \sup_{x\in E} \lvert f_n(x)\rvert$, i.e. by taking the smallest possible values for $M_n$ s.t. $\lvert f_n(x)\rvert \leq M_n$.
	
	\bigskip
	
	[Hint: consider $f_n (x) = \frac{1}{n} \chi_{(n-1,n)}(x)$ where $\chi$ is the indicator function that assumes value $1$ when $x$ is in the specified set, and $0$ otherwise]
\end{Exercise}

\begin{Exercise}[title={***}]


In this exercise we will show that there exists a function that is continuous but nowhere differentiable.

Let $$\phi(x)=\vert x\rvert\;\;\;\;x\in[-1,1]$$
and extend it to all reals $x\in\R$ by imposing
$$\phi(x+2) = \phi(x).$$
Then define $f(x)$ as 
$$f(x)=\sum_{n=0}^{\infty}\left(\frac{3}{4}\right)^n\phi(4^n x).$$
Proceed as follows:
\begin{enumerate}
	\item Show that $\rvert\phi(s)-\phi(t)\lvert \leq \lvert s-t \rvert$.
	\item Show that $\phi$ is continuous.
	\item Find a bound $M_n$ for $\left(\frac{3}{4}\right)^n\phi(4^n x)$ s.t. $\sum_{n=0}^{\infty}M_n$ converges.
	\item show that the series of $f$ converges uniformly so that $f$ is continuous.
	\item Fix $x\in\R$ and introduce the quantity $\delta_m = \frac{1}{2}4^{-m}$ or $\delta_m = -\frac{1}{2}4^{-m}$, where the sign is chosen so that no integer lies between the numbers $4^m x$ and $4^m (x+\delta_m)$. Explain why we can do this.
	\item Show that when $n>m$ then $4^n\delta_m$ is an even integer
	\item Introduce the quantity $$\gamma_n = \frac{\phi(4^n(x+\delta_m))-\phi(4^n x)}{\delta_m}$$ which depends on $m$. Show that for $n>m$ then $\gamma_n=0$, that for $n=m$ then  $\lvert\gamma_n\rvert=4^n$, and that for $n<n$ then  $\lvert\gamma_n\rvert\leq4^n$.
	\item Show that for a sequence $a_n$ s.t. $\lvert a_n\rvert\leq1$ and $\lvert a_m\rvert =1$ then $$\left\lvert\sum_{n=0}^m 3^n a_n\right\rvert \geq 3^m - \sum_{n=0}^{m-1} 3^n$$
	\item Show that $$3^m - \sum_{n=0}^{m-1} 3^n = \frac{1}{2}(3^m+1)$$ by using the properties of geometric series.
	\item Conclude that $$\left\lvert\frac{f(x+\delta_m)-f(x)}{\delta_m}\right\rvert\to +\infty$$ for ${n\to+\infty}$, by using the points above.
	
\end{enumerate}

[Hint: this is Theorem 7.18 of Rudin]
\end{Exercise}

\begin{Exercise} [title={*$\dagger$}]
	Show that if we define a sequence with a double index as $$s_{m,n} = \frac{m}{n+m}$$ then
	$$\lim_{m\to\infty}\lim_{n\to\infty} s_{m,n} \neq \lim_{n\to\infty}\lim_{m\to\infty} s_{m,n}.$$
\end{Exercise}

\begin{Exercise} [title={*$\dagger$}]
	Show that if we define a sequence of continuous functions as $$f_n(x) = \frac{x^2}{(1+x^2)^n}$$ then the function $f$ defined by the sum
	$$f(x) = \sum_{n=0}^{+\infty}f_n(x)$$
	is not continuous.
\end{Exercise}

\begin{Exercise} [title={*$\dagger$}]
	Show that if we define a sequence of differentiable functions as $$f_n(x) = \frac{\sin(nx)}{\sqrt n}$$ then the function $f$ defined by the limit
	$$f(x) = \lim_{n\to+\infty} f_n(x)$$
	is differentiable but $$f'(x)\neq \lim_{n\to+\infty} f_n'(x).$$
\end{Exercise}

\begin{Exercise} [title={*$\dagger$}]
	Show that if we define a sequence of Riemann-integrable functions (in the interval $(0,1)$) as $$f_n(x) = n^2x(1-x^2)^n$$ then the function $f$ defined by the limit
	$$f(x) = \lim_{n\to+\infty} f_n(x)$$
	is also Riemann-integrable on the same interval but $$\int_0^1 f(x)dx\neq \lim_{n\to+\infty} \int_0^1 f_n(x) dx.$$
\end{Exercise}

\begin{Exercise}[title=*]
	Let $(X,d)$ be a metric space. Show that the distance induced by the supremum norm (as defined in Defn. 7.14 of Rudin) on the set of continuous and bounded functions $\mathcal{C}(X)$ satisfies the axioms of distance.
\end{Exercise}

\begin{Exercise}[title={*}]
	Argue that every function of an equicontinuous family is uniformly continuous. Construct a family of uniformly continuous functions that is not an equicontinuous family.
\end{Exercise}


\end{document}

