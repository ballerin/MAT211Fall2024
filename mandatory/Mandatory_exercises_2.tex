\def\outputtype{s}  %s: serie;   c: correction

\documentclass[12pt,a4paper]{article}
%\input{ serieStyleMod}
\usepackage{serieStyle}
\usepackage{graphicx}
\usepackage{verbatim}
\usepackage{float}
\usepackage[most]{tcolorbox}

\newtheorem{theorem}{Theorem}[section]
\newtheorem*{theorem*}{Theorem}
\newtheorem{definition}[theorem]{Definition}
\newtheorem*{definition*}{Definition}
\newtheorem{notation}[theorem]{Notation}
\newtheorem{proposition}[theorem]{Proposition}
\newtheorem{lemma}[theorem]{Lemma}
\newtheorem{remark}[theorem]{Remark}
\newtheorem*{remark*}{Remark}
\newtheorem{remarks}[theorem]{Remarks}
\newtheorem*{remarks*}{Remarks}
\newtheorem{corollary}[theorem]{Corollary}
\newtheorem{ass}[theorem]{Assumption}
\newtheorem*{notation*}{Notation}
\newtheorem{ex}[theorem]{Example}
\newtheorem*{ex*}{Example}
\newtheorem{exs}[theorem]{Examples}
\newtheorem*{exs*}{Examples}
\newtheorem{app}[theorem]{Application}
\newtheorem*{app*}{Application}
\newtheorem{conjecture*}{Conjecture}

\newtcbtheorem{Definition}{Definition}{
	enhanced,
	sharp corners,
	attach boxed title to top left={
		yshifttext=-1mm
	},
	colback=white,
	colframe=gray!75!black,
	fonttitle=\bfseries,
	boxed title style={
		sharp corners,
		size=small,
		colback=gray!75!black,
		colframe=gray!75!black,
	} 
}{def}

\newtcbtheorem{Theorem}{Theorem}{
	enhanced,
	sharp corners,
	attach boxed title to top left={
		yshifttext=-1mm
	},
	colback=white,
	colframe=gray!75!black,
	fonttitle=\bfseries,
	boxed title style={
		sharp corners,
		size=small,
		colback=gray!75!black,
		colframe=gray!75!black,
	} 
}{thm}


\newcommand{\IE}{\mathbb{E}}
\newcommand{\bigo}{\mathcal{O}}
\newcommand{\Headperso}[2]{%
\setcounter{isolution}{0}
\noindent 
Real Analysis \hfill  #2\\
Francesco Ballerin \hfill Universitetet i Bergen
\vskip2ex
\hrule
\vskip2ex
\vskip2ex
\begin{center}\large
#1
\end{center}
\vskip2ex
}

\newcommand{\Tailperso}[1]{%
\vfill	
\vskip2ex\noindent
\hrule
\vskip2ex\noindent
{\small
Informations générales, séries, voir \url{https://moodle.unige.ch/course/view.php?id=5641}}
\vskip2ex\noindent
\vskip2ex
}

\begin{document}

\Headperso{\textbf{Mandatory homework 2}\\ Deadline: October 14, 2024 (at 23:59)}{MAT211 Fall2024}

\noindent


\bigskip
\bigskip


\textbf{Exercise 1 - The Cantor set}\\
Let $C_0=[0,1]$. We build $C_{n+1}$ by induction by cutting each interval of $C_n$ into 3 equal parts (subintervals), and by keeping only the first and last subintervals. For instance, one finds
$$C_1=\left[0,\frac{1}{3}\right]\cup\left[\frac{2}{3},1\right],\quad
C_2=\left[0,\frac{1}{9}\right]\cup\left[\frac{2}{9},\frac{1}{3}\right]\cup\left[\frac{2}{3},\frac{7}{9}\right]\cup\left[\frac{8}{9},1\right].
$$

$(C_n)_n$ is then a decreasing sequence of sets, i.e. $C_{n+1}\subset C_n$, and the Cantor set is defined to be
$$C=\bigcap_{n=0}^\infty C_n.$$
It was introduced by Georg Cantor in 1883, and it is one of the most famous fractals.
\begin{enumerate}
\item Draw the first iterations of $C_n$, for $n=0,1,2,3,4$. Show that $C\neq \emptyset$.
\item Show that $C$ is bounded and closed. Deduce that $C$ is compact by using the appropriate theorems.
\item Given an interval $I=[a,b]$ (but this also work for intervals of the type $[a,b)$, $(a,b]$, $(a,b)$), we define the length of $I$ as $l(I)=b-a$. We extend this definition to a pairwise non-intersecting union of intervals (a set constructed as a union of intervals with no overlap) as the sum of the lengths of the subintervals. For instance, $l(C_1)=\frac{2}{3}$ and $l(C_2)=\frac{4}{9}$.
Show that
$$l(C_n)=\frac{2^{n}}{3^n}.$$
\item
Deduce that $l(C)=0$.
\item Using the previous point, show that $C$ is not open. 

[Hint: an open ball in $\R$ is an interval of the type $(a,b)$ for some $a,b\in\R$.]
\item Show that $C$ does not contain any isolated point, or equivalently that all points of $C$ are limit points.
\end{enumerate}


\solution{

}






\newpage



\noindent\textbf{Exercise 2 - Connected, path-connected, convex and star-shaped spaces}\\



\begin{Definition}{Connected set}{}
	Let $(X,d)$ be a metric space. We say that two sets $A,B\subseteq X$ are \textbf{separated} if both $A \cap \bar B = \emptyset$ and $\bar A \cap B = \emptyset$, i.e. if no point of $A$ lies in the closure of $B$ and no point of $B$ lies in the closure of $A$. A set $E\subseteq X$ is \textbf{connected} if it cannot be written as the union of two nonempty separated sets.
\end{Definition}

\begin{Definition}{Relative topology}{}
	Let $(X,d)$ be a metric space. We say that a set $A\subseteq E$ is \textbf{open-relative} w.r.t. $E$ if for every point $p\in A$ $\exists \varepsilon>0$ s.t. $B_\varepsilon(p)\cap E \subseteq A$. We say that a set $B\subseteq E$ is \textbf{closed-relative} w.r.t. $E$ if it can be written as the complement (in $E$) of an open relative set, i.e. if $\exists A\subseteq E$ open-relative s.t. $B=E\backslash A$. 
\end{Definition}


\begin{Definition}{Path-connected set}{}
	Let $(X,d)$ be a metric space. We say that a subset $E\subset X$ is \textbf{path-connected} (or connected by paths) if for any two points $x,y\in E$ there exists a continuous function $\gamma:[0,1]\to E$ s.t. $\gamma(0)=x$ and $\gamma(1)=y$.
\end{Definition}

\begin{Definition}{Convex set}{}
		Let $(\R^n,\lVert\cdot\rVert)$ be the Euclidean space. We say that a subset $E\subset \R^n$ is \textbf{convex} if for any two points $\textbf{x},\textbf{y}\in E$ and real number $\lambda\in[0,1]$ then $\lambda\textbf{x}+ (1-\lambda)\textbf{y}\in E$.
\end{Definition}

\begin{Definition}{Star-shaped set}{}
	Let $(\R^n,\lVert\cdot\rVert)$ be the Euclidean space.  We say that a subset $E\subset \R^n$ is \textbf{star-shaped} if there is a point $\textbf{x}\in E$ s.t. for any other point $\textbf{y}\in E$ and $\lambda\in[0,1]$ then \[\lambda\textbf{x}+ (1-\lambda)\textbf{y}\in E.\]
\end{Definition}

\begin{enumerate}
	\item Provide two examples of connected and non-connected sets in $\mathbb R^n$ according to Definition 1.
	\item Show that a subset $E\subseteq \mathbb R$ is connected if and only if it has the following property:
	\[x,y,z\in \mathbb R,\;\;\; x,y\in E \text{ and } x<z<y \implies z\in E.\]
	\item Show that a set $B\subseteq E$ is open-relative w.r.t $E\subseteq X$ if and only if there exists an open set $A\subseteq X$ (open w.r.t. $X$) s.t. $A\cap E = B$. Show also that a set $B\subseteq E$ is closed-relative w.r.t $E\subseteq X$ if and only if there exists a closed set $A\subseteq X$ (closed w.r.t. $X$) s.t. $A\cap E = B$.
	\item Show that a subset $E$ of a metric space $(X,d)$ is disconnected (not connected) if and only if it can be written as the union of two disjoint non-empty open sets relatively to $E$ (open in relative topology w.r.t. $E$) .
	\item Show that if $f$ is a continuous map from a metric space to another, then the image through $f$ of any connected set in the domain is connected in the codomain.
	\item Using the previous point show that a non-empty set $E\subset X$ is connected (in the usual sense) if and only if all continuous maps $h:E\to \{0,1\}\subseteq (\R,|\cdot|)$ (maps from $E$ that admits either 0 or 1 as values, with Euclidean metric in the codomain) are constant maps.
	\item Show that if a set is path connected then it is also connected, but that the converse is not true (provide a counterexample).
	\item Show that if a set is convex then it is path-connected, but that the converse is not true (provide a counterexample).
	\item Show that if a set is star-shaped then it is path-connected, but that the converse is not true (provide a counterexample).
\end{enumerate}
\bigskip






%\Tailperso
 
\end{document}

