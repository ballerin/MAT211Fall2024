\def\outputtype{s}  %s: serie;   c: correction

\documentclass[12pt,a4paper]{article}
%\input{ serieStyleMod}
\usepackage{serieStyle}

\usepackage{verbatim}
\usepackage{float}

\newtheorem{theorem}{Theorem}[section]
\newtheorem*{theorem*}{Theorem}
\newtheorem{definition}[theorem]{Definition}
\newtheorem*{definition*}{Definition}
\newtheorem{notation}[theorem]{Notation}
\newtheorem{proposition}[theorem]{Proposition}
\newtheorem{lemma}[theorem]{Lemma}
\newtheorem{remark}[theorem]{Remark}
\newtheorem*{remark*}{Remark}
\newtheorem{remarks}[theorem]{Remarks}
\newtheorem*{remarks*}{Remarks}
\newtheorem{corollary}[theorem]{Corollary}
\newtheorem{ass}[theorem]{Assumption}
\newtheorem*{notation*}{Notation}
\newtheorem{ex}[theorem]{Example}
\newtheorem*{ex*}{Example}
\newtheorem{exs}[theorem]{Examples}
\newtheorem*{exs*}{Examples}
\newtheorem{app}[theorem]{Application}
\newtheorem*{app*}{Application}
\newtheorem{conjecture*}{Conjecture}

\newcommand{\IE}{\mathbb{E}}
\newcommand{\bigo}{\mathcal{O}}
\newcommand{\Headperso}[2]{%
\setcounter{isolution}{0}
\noindent 
Real Analysis \hfill  #2\\
Francesco Ballerin \hfill Universitetet i Bergen
\vskip2ex
\hrule
\vskip2ex
\vskip2ex
\begin{center}\large
#1
\end{center}
\vskip2ex
}

\newcommand{\Tailperso}[1]{%
\vfill	
\vskip2ex\noindent
\hrule
\vskip2ex\noindent
{\small
Informations générales, séries, voir \url{https://moodle.unige.ch/course/view.php?id=5641}}
\vskip2ex\noindent
\vskip2ex
}

\begin{document}

\Headperso{\textbf{Mandatory homework 1}\\ Deadline: September 25, 2023 (at 23:59)}{MAT211 Fall2023}

\noindent
\textit{This mandatory homework consists of three exercises. They need time to be solved properly, so start to work on them well in advance. I strongly advise you to try the exercises on your own, then compare your solutions with the other students, and to ask questions to either Torunn or me if you are really stuck. Try to be as \textbf{rigorous and clear} as possible. If the exercise consists of finding a solution, then finding and stating the solution is not sufficient, and will give you zero points: you have to properly write the procedure to achieve the solution in terms of mathematical statements.
This homework is designed as a training for the exam, and is a \textbf{mandatory} part of the course.
The outcomes of the assignment are pass/fail, with a threshold of 50\%. Try to complete as much of the assignment as possible, as only correct parts of the work count towards the 50\%. Failure to pass this mandatory assignment will automatically exclude a candidate from the final exam.}

\bigskip
\bigskip






\textbf{Exercise 1 - Supremum and infimum}\\
We study the following sets:
$$A=\left\{\frac{(-1)^n}{n}\;\Big|\; n\in \N^*\right\},\quad
B=\left\{\frac{1}{n}\;\Big|\;n\in \N^*\right\}.$$
Do the suprema/infima of $A$ and $B$ exist? If yes, compute them and explain if they also are maxima/minima.

\solution{The sets are not empty, have an upper bound $1$ and the lower bound $-1$, so the suprema/infima exist.
$$
\max(A)=\frac{1}{2}\quad
\min(A)=-1,\quad
\max(B)=1,\quad
\inf(B)=0.
$$
}






\bigskip



\textbf{Exercise 2 - Another definition of $\C$}\\
Let $M$ be the following set of matrices:
$$M:=\left\{\begin{bmatrix}a&b\\-b&a\end{bmatrix}\;\Big|\; a,b\in\mathbb{R}\right\}$$
Endow this set with two operations $+_M$ and $\times_M$ (the classic sum and product of matrices):
\begin{itemize}
\item $\begin{bmatrix}a_1&b_1\\-b_1&a_1\end{bmatrix}+_M\begin{bmatrix}a_2&b_2\\-b_2&a_2\end{bmatrix} = \begin{bmatrix}a_1+a_2&b_1+b_2\\-b_1-b_2&a_1+a_2\end{bmatrix}$

\item $\begin{bmatrix}a_1&b_1\\-b_1&a_1\end{bmatrix}\times_M\begin{bmatrix}a_2&b_2\\-b_2&a_2\end{bmatrix} = \begin{bmatrix}a_1 a_2-b_1 b_2&a_1 b_2 + b_1 a_2\\-b_1a_2-a_1b_2&-b_1 b_2+a_1 a_2\end{bmatrix}$
\end{itemize}

Show that $(M, +_M, \times_M)$ is a field.
\bigskip

Consider now the field of complex numbers $\mathbb{C}$ endowed with its classic operations $+_\mathbb{C}$ and $\times_\mathbb{C}$. Is $(M, +_M, \times_M)$ isomorphic to $(\mathbb C,+_\mathbb{C},\times_\mathbb{C})$?
\bigskip

Hint: to check that two fields are isomorphic you need to
\begin{itemize}
\item find a bijective map $\varphi\colon M\rightarrow\C$,
\item prove that the operations are compatible with $\varphi$:
$$\varphi(A+_M B)=\varphi(A)+_\mathbb{C} \varphi(B),\qquad \varphi(A\times_M B)=\varphi(A)\times_\mathbb{C} \varphi(B).$$
\end{itemize}



\bigskip



\textbf{Exercise 3 - A metric space}\\
For $p,q\in \R$, define
$$d(p,q)=\left\{
\begin{array}{l}
1 \text{ if } p\neq q\\
0 \text{ if } p= q
\end{array}\right.$$
Prove that $(\R,d)$ is a metric space. Find all the open and closed subsets for this topology. Find the compact subsets as well.
























%\begin{center}
%\textbf{{\large Additive subgroups of $\R$}}
%\end{center}
%
%We recall the definition of an additive subgroup of $\R$.
%\begin{definition*}
%A set $G\subset \R$ is an additive subgroup of $\R$ if
%\begin{itemize}
%\item $0\in \R$,
%\item if $x,y\in G$, $x+y\in G$,
%\item if $x\in G$, $-x\in G$.
%\end{itemize}
%\end{definition*}
%The purpose of the exercise is to study the subgroups $(G,+)$ of $(\R,+)$ using topology, and to present applications.
%
%\paragraph{Question 1.}
%We denote $\alpha\Z=\{\alpha k,k\in \Z\}$, where $\alpha\in \R$. Show that $\alpha\Z$ is an additive subgroup of $\R$.
%
%
%\paragraph{Question 2.}
%Let $G\neq \{0\}$ be an additive subgroup of $\R$, and let $\EE=\{x>0, x\in G\}$. Show that $\alpha=\inf \EE$ exists.
%
%
%\paragraph{Question 3.}
%Assume that $\alpha>0$. Show that $\alpha\in G$. Deduce that $G=\alpha \Z$.
%
%
%\paragraph{Question 4.}
%Assume that $\alpha=0$. Show that $G$ is dense in $\R$.
%
%
%We thus proved the following result.
%\begin{theorem*}
%Let $G$ be an additive subgroup of $\R$, then $G=\alpha\Z$ or $G$ is dense in $\R$.
%\end{theorem*}
%
%
%\paragraph{Question 5.}
%Let $a,b\in \R^*$ such that $\frac{a}{b}\notin \Q$. Show that $a\Z+b\Z$ is dense in $\R$. Deduce that $\Z+2\pi\Z$ is dense in $\R$.
%
%
%\paragraph{Question 6.}
%Show that $\{\cos(n), n\in \N\}$ is dense in $[-1,1]$.
%
%
%
%
%
%
%
%
%
%\newpage
%
%\begin{center}
%\textbf{{\large Polar decomposition}}
%\end{center}










%\Tailperso
 
\end{document}

